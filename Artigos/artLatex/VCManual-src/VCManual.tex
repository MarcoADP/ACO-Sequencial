%%%%%%%%%%%%%%%%%%%%%%%%%%%%%%%
%%
%% Package `VauCanSon-G'  version 0.4
%%
%% This is file `VCManual-080612.tex'.
%%
%% IMPORTANT NOTICE:
%%
%% Copyright (C) 2002-2008 Sylvain Lombardy and Jacques Sakarovitch
%%
%% This package may be distributed under the terms of the LaTeX Project
%% Public License, as described in lppl.txt in the base LaTeX distribution.
%% Either version 1.0 or, at your option, any later version.
%%
%% DESCRIPTION:
%%
%%   `VCManual-080612.tex' is the tex file for VauCanSon-G user's and 
%%   reference manual. 
%%  
%%%%%%%%%%%%%%%%%%%%%%%%%%%%%
\documentclass[11pt,twoside]{article}
%
\textheight=220mm  % mieux vaut mettre ces
\textwidth=150mm   % valeurs en ``parametres''
%%%%% output control flags 
\newif\ifdraft  \draftfalse % \drafttrue  %% visualise columns
\newif\ifital \italfalse % \italtrue %% "italian" printing
%%%%% package calls
\usepackage{latexsym,ifthen}
\usepackage{theorem}
\usepackage{xspace}        
\usepackage{amsmath,amsfonts,amscd,amssymb}
\usepackage{makeidx}
\usepackage{subfigure}   
\usepackage{vaucanson-g}
%%%%%%%%%%%%%%%%%%%%%%%%%%%%%%%%%%%%%%%%%%%%%%
%    layout
\topmargin=297mm
\advance \topmargin by -\textheight 
\divide  \topmargin by 2
\advance \topmargin by -1in 
%
\advance \topmargin by -\headheight
\advance \topmargin by -\headsep
%
\leftmargin=210mm
\advance \leftmargin by -\textwidth 
\divide  \leftmargin by 2
\advance \leftmargin by -1in 
%
\oddsidemargin=\leftmargin
\evensidemargin=\leftmargin 
%
\abovedisplayskip=3mm 
\belowdisplayskip=3mm 
\abovedisplayshortskip=0mm
\belowdisplayshortskip=2mm 
\unitlength=1mm 
%%%%%
\addtolength{\parskip}{0.3\baselineskip}
\renewcommand{\baselinestretch}{1.1} 
%%%%%%%%%%%%%%%% colonne + figure %%%%%%%%%%%%%%%%%
\newlength{\ExtendedTextWidth} % largeur du dŽbordement
%                              % pour les figure qui dŽbordent
\newlength{\ColoText}% largeur de la colonne "de texte"
\newlength{\ColoFigu}% largeur de la colonne "de figure"
\newlength{\fboxseptemp} % pour pouvoir restituerla valeur normale
\setlength{\fboxseptemp}{\fboxsep}% parce qu'on va coller le cadre
%                        % pour le draft
\newlength{\parindenttemp} % pour mettre une indentation 
%                          % dans les minipages
\setlength{\parindenttemp}{\parindent}
%\setlength{\fboxsep}{\fboxseptemp}
\setlength{\ExtendedTextWidth}{\marginparwidth}    
% \addtolength{\ExtendedTextWidth}{\marginparsep} % \retraita 
%%%%%%%%%%%%%%%%%%%%%%%%%%%%%%%%%%%%%%
\newcommand{\TextFigu}[3][.66]{%
% \setlength{\fboxsep}{0pt}% mod 000912
%      annulŽ seulement en draft
\setlength{\ColoText}{#1\textwidth}%    
\setlength{\ColoFigu}{\textwidth}%    
\addtolength{\ColoFigu}{-\ColoText}%    
\addtolength{\ColoText}{-.6em}%    
\addtolength{\ColoFigu}{-.6em}%    
\noi
\ifdraft
\setlength{\fboxsep}{0pt}% mod 000912
\fbox{%
\begin{minipage}[t]{\ColoText}
\setlength{\parindent}{\parindenttemp}%
        \par\vspace*{0mm}% pour l'alignement sur le haut
        #2
\end{minipage}}%
\hspace*{1.2em}% 
\fbox{%
\begin{minipage}[t]{\ColoFigu}%
\par\vspace*{0mm}%
#3%
\end{minipage}}
\else
\begin{minipage}[t]{\ColoText}
\setlength{\parindent}{\parindenttemp}%
        \par\vspace*{0mm}% pour l'alignement sur le haut
        #2
\end{minipage}% 
\hspace*{1.2em}% 
\begin{minipage}[t]{\ColoFigu}%
\par\vspace*{0mm}% pour l'alignement sur le haut
#3%
\end{minipage}%
\fi%
\setlength{\fboxsep}{\fboxseptemp}%
}
%%%%%%%%%%%%%%%%%%%%%%%%%%%%%%%%%%%%%%%%%%%
% Modifications pour le header et le footer 
\makeatletter 
\def\ps@jsheadings%
{\let\@mkboth\@gobbletwo
\def\@oddhead{\hbox{} {\jsoddheadleft}   \hfil 
                      {\jsoddheadcenter} \hfil 
                      {\jsoddheadright}  \hbox{}}%  
\def\@oddfoot{\hbox{} {\jsoddfootleft}   \hfil 
                      {\jsoddfootcenter} \hfil 
                      {\jsoddfootright}  \hbox{}}%
\def\@evenhead{\hbox{} {\jsevenheadleft}   \hfil 
                      {\jsevenheadcenter} \hfil 
                      {\jsevenheadright}  \hbox{}}%
\def\@evenfoot{\hbox{} {\jsevenfootleft}   \hfil 
                      {\jsevenfootcenter} \hfil 
                      {\jsevenfootright}  \hbox{}}%
}
\makeatother
%
\newcommand{\jsoddheadleft}{\makebox[0pt][l]{\teteimpgauche}}
\newcommand{\jsoddheadcenter}{\makebox{\teteimpcentre}} 
\newcommand{\jsoddheadright}{\makebox[0pt][r]{\teteimpdroit}}   
\newcommand{\jsoddfootleft}{\makebox[0pt][l]{\piedimpgauche}}   
\newcommand{\jsoddfootcenter}{\makebox{\piedimpcentre}} 
\newcommand{\jsoddfootright}{\makebox[0pt][r]{\piedimpdroit}}  
\newcommand{\jsevenheadleft}{\makebox[0pt][l]{\tetepairgauche}}   
\newcommand{\jsevenheadcenter}{\makebox{\tetepaircentre}} 
\newcommand{\jsevenheadright}{\makebox[0pt][r]{\tetepairdroit}} 
\newcommand{\jsevenfootleft}{\makebox[0pt][l]{\piedpairgauche}}   
\newcommand{\jsevenfootcenter}{\makebox{\piedpaircentre}} 
\newcommand{\jsevenfootright}{\makebox[0pt][r]{\piedpairdroit}} 
%
\newcommand{\teteimpgauche}{}
\newcommand{\teteimpcentre}{{\footnotesize \textsl{\tetedroit}}} 
\newcommand{\teteimpdroit}{}   
\newcommand{\piedimpgauche}{{\scriptsize \piedgauche }}   
\newcommand{\piedimpcentre}{\rm \thepage} 
\newcommand{\piedimpdroit}{{\scriptsize \pieddroit }}  
\newcommand{\tetepairgauche}{}   
\newcommand{\tetepaircentre}{{\footnotesize \textsl{\tetegauche}}} 
\newcommand{\tetepairdroit}{} 
\newcommand{\piedpairgauche}{{\scriptsize \piedgauche }}   
\newcommand{\piedpaircentre}{\rm \thepage} 
\newcommand{\piedpairdroit}{{\scriptsize \pieddroit }} 
%
\newcommand{\tetegauche}{}   % ATTENTION tete de la page de gauche
\newcommand{\tetedroit}{}    % ATTENTION tete de la page de droite
\newcommand{\piedgauche}{}   % ATTENTION gauche du pied de page 
\newcommand{\pieddroit}{}    % ATTENTION droite du pied de page 
%
\pagestyle{jsheadings}
%%%%%%%%%%%%%%%%%%%%%%%%%%%%%%%%%%%%%%%%%%%%%%%%%%%%%
% Modifications pour  sections
\makeatletter
\def\thesection {\arabic{section}}
\makeatother
%%%%%%%%%%%%%%%%%%%%%%%%%%%%%%%%%%%%%%%%%%%%%%%%%%%%%
%%%%%%%%%%%%%%%%%%%%%%%%%%%%%%%%%%%%
% Modifications pour les figures 
\makeatletter
\def\thefigure{\@arabic\c@figure}
\def\fps@figure{ht}
\makeatother
%%%%%%%%%%%%%%%%%%%%%%%%%%%%%%%%%%%%%
%  Macros pour la mise en page
%  espaces de lignes negatifs
\newcommand{\bigskipneg}{\vspace*{-5ex}} %960728
\newcommand{\medskipneg}{\vspace*{-2ex}} %960728
\newcommand{\smallskipneg}{\vspace*{-1ex}} %960728
\newcommand{\miniskipneg}{\vspace*{-0.25ex}} %960728
\newcommand{\skipneg}{\vspace*{-0.5cm}}
% indentation
\newcommand{\noi}{\noindent}
\newlength{\retraita}           % retrait pour i),  ii),  a),  etc.
\setlength{\retraita}{2.4em}    % retrait pour i),  ii),  a),  etc.
%
\newcommand{\jsreta}{\hspace*{-.5\retraita}}
\newcommand{\e}{\text{\quad}}                 % un moins petit espace
\newcommand{\ee}{\text{\qquad}}               % un espace
\newcommand{\eee}{\text{\qquad \qquad}} % et un grand
% separateurs
\newcommand{\tristar}{{\begin{center} * \qquad * \\\ * \ \ \end{center}}}
\newcommand{\jsetoile}{{\begin{center} * \end{center}}}
%%%%%%%%%%%%%%%%%%%%%%%%%%%%%%%%%%%%
%%%%%% \section{abr. latines}
\newcommand{\acontrario}{{\itshape a contrario}\xspace } % 
\newcommand{\ah}{{\itshape ad hoc}\xspace } % pour compatibilitŽ
\newcommand{\adhoc}{{\itshape ad hoc}\xspace }
\newcommand{\apriorinosp}{{\itshape a priori}\xspace } % pas de blanc final 
\newcommand{\apriori}{{\itshape a priori}\xspace } % 
\newcommand{\afortiori}{{\itshape a fortiori}\xspace }
\newcommand{\artcit}{{\itshape art. cit.}\xspace }
\newcommand{\cf}{{\itshape cf.}\xspace }
\newcommand{\Cf}{{\itshape Cf.}\xspace }
\newcommand{\eg}{{\itshape e.g.}\xspace }
\newcommand{\etalii}{{\itshape et~al.}\xspace }
\newcommand{\etc}{{\itshape etc.}\xspace } % pas de blanc laissŽ aprs etc.
\newcommand{\ibid}{{\itshape ibid.}\xspace }
\newcommand{\idem}{{\itshape idem}\xspace }
\newcommand{\ie}{{\itshape i.e.}\xspace }
\newcommand{\infra}{{\itshape infra}\xspace }
\newcommand{\ipsofacto}{{\itshape ipso facto}\xspace }
\newcommand{\itin}{{\itshape in}\xspace }
\newcommand{\modulo}{{\itshape modulo}\xspace }
\newcommand{\mutmat}{{\itshape mutatis mutandis}\xspace } % pas de blanc final
\newcommand{\nb}{\noindent \mbox{\sc nota bene.}\xspace }
\newcommand{\opcit}{{\itshape op. cit.}\xspace }
\newcommand{\opcitnosp}{{\itshape op. cit.}}
\newcommand{\sed}{{\itshape sed}\xspace }
\newcommand{\sqq}{{\itshape sqq.}\xspace }
\newcommand{\supra}{{\itshape supra}\xspace }
\newcommand{\via}{{\itshape via}\xspace }
\newcommand{\vv}{{\itshape vice versa}\xspace } %960728 enlevŽ le blanc final
\newcommand{\verbat}{{\itshape verbatim}\xspace } % pas de blanc final
\newcommand{\jsleq}{\leqslant }
\newcommand{\jsgeq}{\geqslant }
%%%%% tuning of two column macros
\newlength{\jsIndent}% largeur de la colonne "de figure"
\setlength{\jsIndent}{1.8cm}%    
\setlength{\ColoText}{.3\textwidth}%    
\setlength{\ColoFigu}{\textwidth}%    
\addtolength{\ColoFigu}{-\ColoText}%    
\addtolength{\ColoText}{-.6em}%    
\addtolength{\ColoFigu}{-.6em}%    
\addtolength{\ColoText}{\jsIndent}
%
\newlength{\ColSource}% largeur de la colonne "de texte"
\newlength{\ColFigur}% largeur de la colonne "de figure"
\setlength{\ColSource}{.7\textwidth}%    
\setlength{\ColFigur}{\textwidth}%    
\addtolength{\ColFigur}{-\ColSource}%    
\addtolength{\ColSource}{-.6em}%    
\addtolength{\ColFigur}{-.6em}%    
\addtolength{\ColFigur}{\jsIndent}
%
\renewcommand{\tetegauche}{\VCSG  manual}                %ATTN tete de la page de gauche
\renewcommand{\tetedroit}{\VCSG  manual}                %ATTN tete de la page de droite
\renewcommand{\piedgauche}{Version 0.4}%ATTN gauche du pied de page
\renewcommand{\pieddroit}{October 27.,  2008}   %ATTN droite du pied de page
\renewcommand{\piedimpcentre}{--~~\rm \thepage ~~--} 
\renewcommand{\piedpaircentre}{--~~\rm \thepage ~~--} 
%%%%% 
\newcommand{\sep}%
{\begin{center}
       --- $\circ$ ---
 \end{center}} 
\newcommand{\PSTricks}{\texttt{PSTricks}\xspace}
\newcommand{\URLtilde}{${\scriptstyle\pmb{\sim}}$}
\makeindex
%%%%% 
%\ShowFrame
%%%%%%%%%%%%%%%%%%%%%%%%%%%%%%%
\begin{document}
%
\ifital \thispagestyle{empty}\mbox{ }\clearpage%
       \addtocounter{page}{-1}\fi
%       
\title{{\LARGE \VCSG}\\
A package for drawing automata and graphs\\
\smallskip
{\large User's  manual}}
\author{%
\textsl{Sylvain Lombardy}\\[0.5ex]
{\normalsize IGM-LabInfo , Univ. Paris-Est}%
\and 
\textsl{Jacques Sakarovitch}\\[0.5ex]
{\normalsize LTCI, CNRS / TELECOM ParisTech}%
}

\date{October 27.,  2008}
\maketitle 
        
\begin{center}
{\large Version 0.4}
\end{center}
\sep
%
\VCSG is a package that allows to draw automata within 
texts written using \LaTeX.
It is a set of macros that uses commands of the beautiful \PSTricks package,
due to Timothy van Zandt\footnote{%
   The maintenance and evolution of \PSTricks has now been taken over by H.~Voss.}
and which is part of 
the current \LaTeX\ distribution.
\VCSG is the result of the experience gained in composing the several
hundreds of 
figures in the book \emph{El\'ements de th\'eorie des automates} 
(\cite{Sak01}) and in several other papers and conferences slides
of the authors.

The design of \VCSG implements the following
underlying philosophy:
\begin{enumerate}
    \item  `simple' automata should be 
described with simple commands.

    \item Every element of a figure can easily be given a style 
    (which differs from the implicit one).

    \item  The complexity of commands (or the number of things to be remembered 
to use them) should gradually grow with the complexity of the figure 
composed by these commands.

    \item  It should be possible to handle the figures both in size and 
        appearance without modifying them.
\end{enumerate}

The following example\footnote{%
    somewhat artificial but which exhibits most of the simple commands.}
shows how a simple automaton can be drawn 
with commands, in which only the minimal information needed (position 
and label of states, shape and label of transitions) is made explicit.



\noi 
\hspace*{-\jsIndent}
\begin{minipage}[t]{\ColFigur}%
\par\vspace*{0mm}% pour l'alignement sur le haut
\bigskip 
\begin{center}
\scalebox{\VCScale}{%
\begin{VCPicture}{(0,-2)(6,2)}
% states
\State[p]{(0,0)}{A}\State{(3,0)}{B}
\State[r]{(6,0)}{C}
%
\Initial{A}\Final{C}
% transitions 
\EdgeL{A}{B}{a}
\ArcL{B}{C}{b}\ArcL{C}{B}{b}
\LoopN{A}{a}\LoopS{C}{d}
%
\end{VCPicture}%
}%
\end{center}
\end{minipage}%
\hspace*{1.2em}% 
\begin{minipage}[t]{\ColSource}
\setlength{\parindent}{\parindenttemp}%
\par\vspace*{0mm}% pour l'alignement sur le haut
\footnotesize
\begin{verbatim}
\begin{VCPicture}{(0,-2)(6,2)}
% states
\State[p]{(0,0)}{A} \State{(3,0)}{B} \State[r]{(6,0)}{C}
% initial--final
\Initial{A} \Final{C}
% transitions 
\EdgeL{A}{B}{a} \ArcL{B}{C}{b} \ArcL{C}{B}{b}
\LoopN{A}{a} \LoopS{C}{d}
%
\end{VCPicture}
\end{verbatim}
\normalsize
\end{minipage}% 

\bigskip \bigskip

The objectives of \VCSG are achieved by the implicit definition of a 
large number of  
parameters that control the geometry of the figure: size of states, 
width of lines, \etc 
and by the definition of commands that allow to 
handle and modify these parameters.

The following pages gives a presentation of \VCSG that 
gradually introduces the commands from the basic ones that allow 
to draw already a good deal of automata to the more 
sophisticated ones that are used to modify the internal parameters.



\setcounter{tocdepth}{1}
~
{\small
\tableofcontents      
}


\newpage
\addtocounter{section}{-1}
%%%%%%%%%%%%%%%%%%%%%%
\section{Downloading and using the package}\label{sec.d}


The \VCSG package is downloadable from the URL:

\texttt{/igm.univ-mlv.fr/\URLtilde lombardy/Vaucanson-G/}

\noindent
and consists, beside the present manual, in seven files:
the wrapper \texttt{vaucanson-g.sty}, 
the main macro file \texttt{VauCanSon-G.tex}, 
and five other ancillary files\footnote{%
   {\it Cf.} Section~\ref{sec.mod} for an explanation on the role of 
   these files.}
for options, patches and colors:
\texttt{VCPref-default.tex}, 
\texttt{VCPref-slides.tex},
\texttt{VCPref-beamer.tex},
\texttt{VCPref-mystyle.tex},
\texttt{VCColor-names.def}
\index{VCPref-default}%
\index{VCPref-slides}%
\index{VCPref-beamer}%
\index{VCPref-mystyle}%
\index{VCColor-names.def}%
that should all be stored in the user's \texttt{texinputs} directory.

A \LaTeX\ file that will use \VCSG macros should contain the call to 
the wrapper in the preamble:

\verb+\usepackage{vaucanson-g}+ 

The \VCSG commands are to be used in a \LaTeX\ file and not in a plain \TeX\ 
file for it uses the macro \verb+\newcommand+ that does not exist in \TeX.

\VCSG is based on the \PSTricks package.
The knowledge of this macro package,
described in \cite{Girou94}, \cite[Ch.~4]{GRM97}, 
\cite{GRMRV07}, \cite{Voss07},
is not necessary to use the \VCSG commands.\footnote{%
   It probably helps in understanding how they are built, and is 
   useful if one wants to make figures that consists of automata but 
   also contains other graphical elements
   ({\it cf.} Section~\ref{sec.pla}).}
On the contrary, \VCSG has somehow be written to save on it.
In particular, the \VCSG commands follow the simple (and rigid) 
syntax of \LaTeX\ commands concerning the way the arguments are 
specified, and not the much more flexible syntax of \PSTricks itself.


But the call to \PSTricks commands has the consequence that \VCSG 
commands generate pieces of \texttt{Postscript} files which in turn 
has two main outcomes\footnote{%
    which a user should have in mind before sending complaint and 
    request.}:
    \begin{itemize}
        \item  \VCSG is not compatible with \verb+pdflatex+ ;
    
        \item  the figures produced with \VCSG are not likely to be 
        seen on a \texttt{DVI} viewer: the space occupied by the figure 
        will appear empty and, in most cases, the labels will appear 
        at the bottom of it.
    \end{itemize}

% %%%%%%%%%%%%%%%%%%%ù
% \section{Using Beamer document class}
\VCSG can be used with the \texttt{beamer} document class to design slides.
The document class
\index{beamer class}%
should be called with {\tt xcolor=pst,dvips} options:

\verb+\documentclass[xcolor=pst,dvips]{beamer}+.

\noindent
Thus equipped, we may begin.
%%%%%%%%%%%%%%%%%%%%%%
\section{Basic commands}\label{sec.b}

An automaton is a \emph{labelled directed graph} where some of the vertices, 
called \emph{states}, are distinguished to be \emph{initial} or 
\emph{final} states.
The basic commands allow then \emph{to define} and \emph{to draw} 
states, to give them the quality of being initial or final and to give 
them \emph{labels} as well.
They allow to draw \emph{transitions} 
between states (that have been defined before);
there are 3 kinds of transitions: \emph{(straight) edges}, \emph{arcs} and 
\emph{loops}.

These commands have to be put in an environment that defines \emph{a 
picture}.
All graphical parameters of these elements (size, line width, \etc ) 
are kept implicit when using these basic commands.

%%%%%%%%%%%%%%%%%%%%%%
\subsection{The \texttt{VCPicture} environment}\label{subsec.pic}
\index{VCPicture}

\noi 
\hspace*{-\jsIndent}
\begin{minipage}[t]{\ColoText}
        \par\vspace*{0mm}% pour l'alignement sur le haut
        \footnotesize
\verb+\begin{VCPicture}{+($x_{0}$, $y_{0}$)($x_{1}$, $y_{1}$)\verb+}+

\ldots

\verb+\end{VCPicture}+
\normalsize
\end{minipage}% 
\hspace*{1.2em}% 
\begin{minipage}[t]{\ColoFigu}%
% \setlength{\parindent}{\parindenttemp}%
\par\vspace*{0mm}% pour l'alignement sur le haut
Defines a box that ``encloses'' the elements defined in the 
environment.
Its lower left (resp. upper right) corner has 
coordinates~($x_{0}$, $y_{0}$) (resp.~($x_{1}$, $y_{1}$)).
\end{minipage}%

\medskip

The~$x_{i}$ and~$y_{i}$ are either \emph{length} or \emph{numbers} in 
which case they represent the corresponding length in centimeter (which 
is the implicit length unit used in \VCSG).
The length of the box is thus~$x_{1}-x_{0}$, its 
height~$y_{1}-y_{0}$, its \emph{baseline point} stands at the middle 
of the height.
As usual in \TeX, a picture may well ``contain'' an element that 
actually lays outside of the box.

Note that the environment has only \emph{one} argument consisting of \emph{two} 
point coordinates enclosed in adjacent parentheses.
%%%%%%%%%%%%%%%%%%%%%%
\subsection{The \texttt{State} commands}\label{subsec.sta}
\index{State}

States are represented by circles, inside which an information (the 
\emph{Label}) can be written.
Every state is given a \emph{Name} which is used then internally in the figure to describe the 
transitions that start from it or end at it.

\noi 
\hspace*{-\jsIndent}
\begin{minipage}[t]{\ColoText}
        \par\vspace*{0mm}% pour l'alignement sur le haut
        \footnotesize
\verb+\State[+\textsl{Label}\verb+]{+%
   $(x,y)$\verb+}{+%
   \textsl{Name}\verb+}+
\normalsize
\end{minipage}% 
\hspace*{1.2em}% 
\begin{minipage}[t]{\ColoFigu}%
% \setlength{\parindent}{\parindenttemp}%
\par\vspace*{0mm}% pour l'alignement sur le haut
Defines a state with name \textsl{Name} and draws it 
at the coordinate~$(x,y)$ with label \textsl{Label} written inside.
\end{minipage}%

\medskip 
\noi 
The parameter \textsl{Label} is optional; default value is the empty 
string.
It is written in `\TeX\ mathmode'.

\medskip 
\noi 
\hspace*{-\jsIndent}
\begin{minipage}[t]{\ColoText}
        \par\vspace*{0mm}% pour l'alignement sur le haut
        \footnotesize
\verb+\Initial[+\textsl{Dir}\verb+]{+%
   \textsl{Name}\verb+}+
\normalsize
\end{minipage}% 
\hspace*{1.2em}% 
\begin{minipage}[t]{\ColoFigu}%
% \setlength{\parindent}{\parindenttemp}%
\par\vspace*{0mm}% pour l'alignement sur le haut
Makes the state \textsl{Name} initial, \ie  draws an incoming arrow 
from the direction \textsl{Dir} to state {\sl Name}.
\end{minipage}%
\index{Initial}\index{Final}

\medskip 
\noi 
\hspace*{-\jsIndent}
\begin{minipage}[t]{\ColoText}
        \par\vspace*{0mm}% pour l'alignement sur le haut
        \footnotesize
\verb+\Final[+\textsl{Dir}\verb+]{+%
   \textsl{Name}\verb+}+
\normalsize
\end{minipage}% 
\hspace*{1.2em}% 
\begin{minipage}[t]{\ColoFigu}%
% \setlength{\parindent}{\parindenttemp}%
\par\vspace*{0mm}% pour l'alignement sur le haut
Makes the state \textsl{Name} final, \ie  draws an outgoing arrow 
from state {\sl Name} in the direction \textsl{Dir}.
\end{minipage}%

\medskip 
\noi 
The parameter \textsl{Dir} is optional; the possible values are the 
eight main
compass points: {\sl n}, {\sl s}, {\sl e}, {\sl w}, {\sl ne}, {\sl 
nw}, {\sl se} and {\sl sw}. 
Default value is {\sl w} for \verb+\Initial+, 
{\sl e} for \verb+\Final+.

Many authors, in particular those dealing with automata on infinite words,
rather indicate that a state is final by drawing it with doubleline.
The reason why it is not possible to achieve this result by a sequence of
two commands comes from the \PSTricks macros that are called by the \VCSG
commands.

\medskip
\noi 
\hspace*{-\jsIndent}
\begin{minipage}[t]{\ColoText}
        \par\vspace*{0mm}% pour l'alignement sur le haut
        \footnotesize
\verb+\FinalState[+\textsl{Label}\verb+]{+%
   $(x,y)$\verb+}{+%
   \textsl{Name}\verb+}+
\normalsize
\end{minipage}% 
\hspace*{1.2em}% 
\begin{minipage}[t]{\ColoFigu}%
% \setlength{\parindent}{\parindenttemp}%
\par\vspace*{0mm}% pour l'alignement sur le haut
Same as \verb+\State+ but draws state \textsl{Name} 
with a doubleline which is a common way to indicate that a state is final.
\index{FinalState}
\end{minipage}%
\medskip

\noi 
\hspace*{-\jsIndent}
\begin{minipage}[t]{\ColFigur}%
\par\vspace*{0mm}% pour l'alignement sur le haut
\bigskip 
\begin{center}
\scalebox{\VCScale}{%
\begin{VCPicture}{(0,-2)(6,2)}
% states
\State[p]{(0,1)}{A} \State{(2,-1)}{B} \State[r]{(4,1)}{C}
\FinalState[r]{(6,-1)}{D}
% initial--final
\Initial{A} \Final{A}
\Initial[ne]{B}\Initial[nw]{B}
\Initial[se]{B}\Initial[sw]{B}
\Initial{C} \Final[w]{C} \Initial[n]{D} \Final[s]{D}
%
\end{VCPicture}%
}%
\end{center}
\end{minipage}%
\hspace*{1.2em}% 
\begin{minipage}[t]{\ColSource}
\setlength{\parindent}{\parindenttemp}%
\par\vspace*{0mm}% pour l'alignement sur le haut
\footnotesize
\begin{verbatim}
\begin{VCPicture}{(0,-2)(6,2)}
% states
\State[p]{(0,1)}{A} \State{(2,-1)}{B} \State[r]{(4,1)}{C}
\FinalState[r]{(6,-1)}{D}
% initial--final
\InitialL{A} \Final{A}
\Initial[ne]{B} \Initial[nw]{B}
\Initial[se]{B} \Initial[sw]{B}
\Initial{C} \Final[w]{C} \Initial[n]{D} \Final[s]{D}
%
\end{VCPicture}%
\end{verbatim}
\normalsize
\end{minipage}% 

%\newpage 
%%%%%%%%%%%%%%%%%%%%%%
\subsection{The \texttt{Edge} commands}\label{subsec.edg}

Edges, like the other transitions, are \emph{oriented} from the origin 
state to the end state.
The \emph{left} and the \emph{right} sides of a transition are defined 
with respect to this orientation, being the left and the right hand of 
someone standing on the origin state and following
the transition towards the end state.

\noi 
\hspace*{-\jsIndent}
\begin{minipage}[t]{\ColoText}
        \par\vspace*{0mm}% pour l'alignement sur le haut
\medskip 
        \footnotesize
\verb+\EdgeL[+\textsl{Pos}\verb+]{+%
   \textsl{Start}\verb+}{+%
   \textsl{End}\verb+}{+%
   \textsl{Label}\verb+}+
   
\medskip 
\verb+\EdgeR[+\textsl{Pos}\verb+]{+%
   \textsl{Start}\verb+}{+%
   \textsl{End}\verb+}{+%
   \textsl{Label}\verb+}+
   \normalsize
\end{minipage}% 
\hspace*{1.2em}% 
\begin{minipage}[t]{\ColoFigu}%
% \setlength{\parindent}{\parindenttemp}%
\par\vspace*{0mm}% pour l'alignement sur le haut
Draws a straight transition from the state \textsl{Start} to the 
state \textsl{End} with the label \textsl{Label} on the left side
for \verb+\EdgeL+, on the right side for \verb+\EdgeR+.\index{Edge}
\end{minipage}%

\medskip 
\noi 
The parameter {\sl Pos} is optional. 
It is a number ($0\jsleq \text{{\sl Pos}} \jsleq 1$) that defines the 
position of the label on the edge.
Default value is~$.45$ and may be used most of the time ---
this slight disymmetry gives some ``dynamic'' to the figure.
Explicit value for {\sl Pos} is used to avoid collision with other 
elements of the figure. 
As for states, the label is written in `\TeX\ mathmode'.

\noi 
\hspace*{-\jsIndent}
\begin{minipage}[t]{\ColFigur}%
\par\vspace*{0mm}% pour l'alignement sur le haut
\bigskip 
\begin{center}
\scalebox{\VCScale}{%
\begin{VCPicture}{(0,-2)(6,2)}
% states
\State[A]{(0,0)}{A} \State[B]{(2,-2)}{B} \State[C]{(4,2)}{C}
\State[D]{(6,0)}{D}
% transitions 
\EdgeR{A}{B}{a} \EdgeL{A}{C}{b} \EdgeR{B}{D}{c} 
\EdgeL{C}{D}{d} \EdgeL{C}{B}{} \EdgeL[.3]{A}{D}{x}
%
\end{VCPicture}%
}%
\end{center}
\end{minipage}%
\hspace*{1.2em}% 
\begin{minipage}[t]{\ColSource}
\setlength{\parindent}{\parindenttemp}%
\par\vspace*{0mm}% pour l'alignement sur le haut
\footnotesize
\begin{verbatim}
\begin{VCPicture}{(0,-2)(6,2)}
% states
\State[A]{(0,0)}{A} \State[B]{(2,-2)}{B} \State[C]{(4,2)}{C}
\State[D]{(6,0)}{D}
% transitions 
\EdgeR{A}{B}{a} \EdgeL{A}{C}{b} \EdgeR{B}{D}{c} 
\EdgeL{C}{D}{d} \EdgeL{C}{B}{} \EdgeL[.3]{A}{D}{x}
%
\end{VCPicture}
\end{verbatim}
\normalsize
\end{minipage}% 

%%%%%%%%%%%%%%%%%%%%%%
\subsection{The \texttt{Arc} commands}\label{subsec.arc}

Arcs are transitions that link the {\sl Start} state to the {\sl End} 
state with a curved line.
The main parameter of such a curve is the \emph{angle} made at both 
ends by the curve with the straight line that goes from  {\sl Start} to 
{\sl End} and it can be oriented to the \emph{left} or to the 
\emph{right}.
In the sequel, the letter \texttt{\textsl{X}} stands for either \verb+L+ 
or \verb+R+. 

\noi 
\hspace*{-\jsIndent}
\begin{minipage}[t]{\ColoText}
        \par\vspace*{0mm}% pour l'alignement sur le haut
\medskip 
        \footnotesize
\verb+\Arc+\texttt{\textsl{X}}\verb+[+\textsl{Pos}\verb+]{+%
   \textsl{Start}\verb+}{+%
   \textsl{End}\verb+}{+%
   \textsl{Label}\verb+}+
   
\medskip 
\verb+\LArc+\texttt{\textsl{X}}\verb+[+\textsl{Pos}\verb+]{+%
   \textsl{Start}\verb+}{+%
   \textsl{End}\verb+}{+%
   \textsl{Label}\verb+}+
   \normalsize
\end{minipage}% 
\hspace*{1.2em}% 
\begin{minipage}[t]{\ColoFigu}%
% \setlength{\parindent}{\parindenttemp}%
\par\vspace*{0mm}% pour l'alignement sur le haut
Same syntax as \verb+\Edge+\texttt{\textsl{X}}.\index{Arc}\index{LArc} 

The starting angle is~$15^o$ for \verb+Arc+\texttt{\textsl{X}}, 
$30^o$ for \verb+LArc+\texttt{\textsl{X}}\verb++. 

Parameter {\sl Pos} is optional. 
Default value is~$.4$ .
\end{minipage}%

\medskip 

\noi 
\hspace*{-\jsIndent}
\begin{minipage}[c]{\ColFigur}%
\par\vspace*{0mm}% pour l'alignement sur le haut
\bigskip 
\begin{center}
\scalebox{\VCScale}{%
\begin{VCPicture}{(0,-2)(6,2)}
% states
\State[A]{(0,0)}{A} \State[B]{(2,-2)}{B} \State[C]{(4,2)}{C}
\State[D]{(6,0)}{D}
%initial-final
\Initial{A} \Final[s]{B} \Final[n]{C}
% transitions 
\ArcR{A}{B}{a} \ArcL{A}{C}{b} \LArcR{B}{D}{c} \LArcL{C}{D}{d}
\ArcL{A}{B}{a'} \EdgeR{A}{C}{b'} \EdgeL{B}{D}{c'} \LArcR{C}{D}{d'}
%
\end{VCPicture}%
}%
\end{center}
\end{minipage}%
\hspace*{1.2em}% 
\begin{minipage}[c]{\ColSource}
\setlength{\parindent}{\parindenttemp}%
\par\vspace*{0mm}% pour l'alignement sur le haut
\footnotesize
\begin{verbatim}
\begin{VCPicture}{(0,-2)(6,2)}
% states
\State[A]{(0,0)}{A} \State[B]{(2,-2)}{B} \State[C]{(4,2)}{C}
\State[D]{(6,0)}{D}
%initial-final
\Initial{A} \Final[s]{B} \Final[n]{C}
% transitions 
\ArcR{A}{B}{a} \ArcL{A}{C}{b} \LArcR{B}{D}{c} \LArcL{C}{D}{d}
\ArcL{A}{B}{a'} \EdgeR{A}{C}{b'} \EdgeL{B}{D}{c'}
\LArcR{C}{D}{d'}
%
\end{VCPicture}
\end{verbatim}
\normalsize
\end{minipage}% 

%%%%%%%%%%%%%%%%%%%%%%
\subsection{The \texttt{Loop} commands}\label{subsec.loo}

Loops are transitions that link a state to itself, a state which is 
thus both the origin and the end of the transition.
The two main parameters of such a loop are the \emph{opening angle} 
and the \emph{direction} of the loop.
In the sequel, the letter \texttt{\textsl{Y}} stands for any of the compass 
points: \verb+N+, \verb+S+, \verb+E+, \verb+W+, \verb+NE+, \verb+NW+, 
\verb+SE+ or \verb+SW+.\index{Loop}\index{CLoop} 

\noi 
\hspace*{-\jsIndent}
\begin{minipage}[t]{\ColoText}
        \par\vspace*{0mm}% pour l'alignement sur le haut
\medskip 
        \footnotesize
\verb+\Loop+\texttt{\textsl{Y}}\verb+[+\textsl{Pos}\verb+]{+%
   \textsl{Name}\verb+}{+%
   \textsl{Label}\verb+}+
   
\medskip 
\verb+\CLoop+\texttt{\textsl{Y}}\verb+[+\textsl{Pos}\verb+]{+%
   \textsl{Name}\verb+}{+%
   \textsl{Label}\verb+}+
\end{minipage}% 
\hspace*{1.2em}% 
\begin{minipage}[t]{\ColoFigu}%
% \setlength{\parindent}{\parindenttemp}%
\par\vspace*{0mm}% pour l'alignement sur le haut
Draw a loop around the state {\sl Name} with label {\sl Label}.

The direction of the loop is given by \texttt{\textsl{Y}}.
The opening of the loop is~$60^o$ for the \verb+Loop+\texttt{\textsl{Y}}\verb++ commands,
$44^o$ for the \verb+CLoop+\texttt{\textsl{Y}}\verb++ commands.

Parameter {\sl Pos} is optional. 
Default value is~$.25$ .
\end{minipage}%

\medskip 

\noi 
\hspace*{-\jsIndent}
\begin{minipage}[t]{\ColFigur}%
\par\vspace*{0mm}% pour l'alignement sur le haut
\bigskip 
\begin{center}
\scalebox{\VCScale}{%
\begin{VCPicture}{(0,-3)(7,3)}
% states
\State[A]{(0,0)}{A} \State[B]{(3,-2)}{B} \State[C]{(3,2)}{C}
\State[D]{(6,0)}{D}
% transitions 
\LoopE{A}{e} \LoopW{A}{w} \LoopS{A}{s} \LoopN[.5]{A}{n} 
% \LoopNE{B}{ne} \LoopNW[.5]{B}{nw} 
\LoopSE[.5]{B}{se} \LoopSW[.5]{B}{sw} 
\CLoopNE[.6]{C}{ne} \CLoopNW[.6]{C}{nw} 
% \CLoopSE[.5]{C}{se} \CLoopSW[.5]{C}{sw} 
\CLoopE{D}{e} \CLoopW{D}{w} \CLoopS{D}{s} \CLoopN[.5]{D}{n} 
%
\end{VCPicture}%
}%
\end{center}
\end{minipage}%
\hspace*{1.2em}% 
\begin{minipage}[t]{\ColSource}
\setlength{\parindent}{\parindenttemp}%
\par\vspace*{0mm}% pour l'alignement sur le haut
\footnotesize
\begin{verbatim}
\begin{VCPicture}{(0,-3)(7,3)}
% states
\State[A]{(0,0)}{A} \State[B]{(3,-2)}{B} \State[C]{(3,2)}{C}
\State[D]{(6,0)}{D}
% transitions 
\LoopE{A}{e} \LoopW{A}{w} \LoopS{A}{s} \LoopN[.5]{A}{n} 
\LoopSE[.5]{B}{se} \LoopSW[.5]{B}{sw} 
\CLoopNE[.6]{C}{ne} \CLoopNW[.6]{C}{nw} 
\CLoopE{D}{e} \CLoopW{D}{w} \CLoopS{D}{s} \CLoopN[.5]{D}{n} 
%
\end{VCPicture}
\end{verbatim}
\normalsize
\end{minipage}% 



%%%%%%%%%%%%%%%%%%%%%%
\subsection{An example}\label{subsec.ex1}
This is an automaton with multiplicity in $\mathbb{N}$.
The multiplicity of a word~$u$ is the square of the number represented by~$u$
in base~$2$, when~$a$ is interpreted as~$0$ and $b$ as~$1$.



\noi 
\hspace*{-\jsIndent}
\begin{minipage}[t]{\ColFigur}%
\par\vspace*{0mm}% pour l'alignement sur le haut
\bigskip 
\begin{center}
\scalebox{\VCScale}{%
\begin{VCPicture}{(0,-2)(6,2)}
% states
\State[i]{(0,0)}{A} \State[q]{(3,0)}{B} \State[t]{(6,0)}{C}
% initial-final
\Initial{A} \Final{C}
% transitions 
\EdgeR{A}{B}{2b} \EdgeR{B}{C}{2b} \LArcL{A}{C}{b}
\LoopN{A}{a} \LoopS{A}{b} \LoopS[.5]{B}{2a+2b} 
\LoopN[.75]{C}{4a} \LoopS[.75]{C}{4b}
%
\end{VCPicture}%
}%
\end{center}
\end{minipage}%
\hspace*{1.2em}% 
\begin{minipage}[t]{\ColSource}
\setlength{\parindent}{\parindenttemp}%
\par\vspace*{0mm}% pour l'alignement sur le haut
\footnotesize
\begin{verbatim}
\begin{VCPicture}{(0,-2)(6,2)}
% states
\State[i]{(0,0)}{A} \State[q]{(3,0)}{B} \State[t]{(6,0)}{C}
% initial-final
\Initial{A} \Final{C}
% transitions 
\EdgeR{A}{B}{2b} \EdgeR{B}{C}{2b} \LArcL{A}{C}{b}
\LoopN{A}{a} \LoopS{A}{b} \LoopS[.5]{B}{2a+2b} 
\LoopN[.75]{C}{4a} \LoopS[.75]{C}{4b}
%
\end{VCPicture}
\end{verbatim}
\normalsize
\end{minipage}% 


%%%%%%%%%%%%%%%%%%%%%%
\section{Using preset parameters}\label{sec.pre}

Together with the basic commands, \VCSG gives easy access to two convenient
parameters for drawing automata: global scaling and state size.

%%%%%%%%%%%%%%%%%%%%%%
\subsection{Global scaling}\label{subsec.sca}

The meticulous reader has noticed that the actual size, mesured for 
instance by the distance between the states, is not the one given in 
the source code.
This is because we have slightly cheated and the source code of the 
picture is indeed encapsulated in a \verb+\VCDraw+ command that uses 
a preset parameter \verb+\VCScale+ as shown below, with no 
cheating anymore.
\index{VCScale}\index{VCDraw}

\ShowFrame
\noi 
\hspace*{-\jsIndent}
\begin{minipage}[t]{\ColFigur}%
\par\vspace*{0mm}% pour l'alignement sur le haut
\medskipneg 
\begin{center}
\begin{VCPicture}{(0,-1)(3,1)}
\State[p]{(0,0)}{A} \State[q]{(3,0)}{B}%
\end{VCPicture}%
\end{center}
\end{minipage}%
\hspace*{1.2em}% 
\begin{minipage}[t]{\ColSource}
\setlength{\parindent}{\parindenttemp}%
\par\vspace*{0mm}% pour l'alignement sur le haut
\footnotesize
\begin{verbatim}
\begin{VCPicture}{(0,-1)(3,1)}
\State[p]{(0,0)}{A} \State[q]{(3,0)}{B}
\end{VCPicture}
\end{verbatim}
\normalsize
\end{minipage}% 

\noi 
\hspace*{-\jsIndent}
\begin{minipage}[t]{\ColFigur}%
\par\vspace*{0mm}% pour l'alignement sur le haut
\medskip 
\begin{center}
\VCDraw{%
\begin{VCPicture}{(0,-1)(3,1)}
\State[p]{(0,0)}{A} \State[q]{(3,0)}{B}%
\end{VCPicture}}%
\end{center}
\end{minipage}%
\hspace*{1.2em}% 
\begin{minipage}[t]{\ColSource}
\setlength{\parindent}{\parindenttemp}%
\par\vspace*{0mm}% pour l'alignement sur le haut
\footnotesize
\begin{verbatim}
\VCDraw{%
\begin{VCPicture}{(0,-1)(3,1)}
\State[p]{(0,0)}{A} \State[q]{(3,0)}{B}%
\end{VCPicture}}
\end{verbatim}
\normalsize
\end{minipage}% 

\medskip
\noi 
Frames that appear on figures below are an option of \VCSG that we
shall see later. They point out the effect of the scale.

\medskip
\noi 
There is no reason indeed for an author to ``program'' the drawing of 
an automaton at the size it will appear on his, or her, 
paper. 
On the contrary, he, or she, is likely to sketch the figure on a 
ruled paper at a larger scale, giving the points to be defined integer 
coordinates.
And then she, or he, will try to put the figure within the text at a 
reasonable size and thus would like to scale it.

The parameter \verb+\VCScale+ contains this global scaling factor and is
initialized to the value \verb+0.6+ , which corresponds to the 
command \verb+\MediumPicture+.
Three other commands: 
\verb+\TinyPicture+ ,
\verb+\SmallPicture+ and
\verb+\LargePicture+
are available, that set \verb+\VCScale+ to the values 
\verb+0.42+ ,
\verb+0.5+ and
\verb+0.85+ respectively.
\index{MediumPicture}\index{TinyPicture}\index{SmallPicture}
\index{LargePicture}

These command have to be used \emph{outside} the scope of the 
\verb+VCPicture+ environment; and this environment should be called 
as the parameter of a \verb+\VCDraw+ .

\noi 
\hspace*{-\jsIndent}
\begin{minipage}[c]{\ColFigur-.7cm}%
\par\vspace*{0mm}% pour l'alignement sur le haut
\begin{center}
\TinyPicture\VCDraw{%
\begin{VCPicture}{(-1,-1)(1,1)}\State[p]{(0,0)}{A}\end{VCPicture}}
\SmallPicture\VCDraw{%
\begin{VCPicture}{(-1,-1)(1,1)}\State[p]{(0,0)}{A}\end{VCPicture}}
\MediumPicture\VCDraw{%
\begin{VCPicture}{(-1,-1)(1,1)}\State[p]{(0,0)}{A}\end{VCPicture}}
\LargePicture\VCDraw{%
\begin{VCPicture}{(-1,-1)(1,1)}\State[p]{(0,0)}{A}\end{VCPicture}}
\end{center}
\end{minipage}%
\hspace*{1.2em}% 
\begin{minipage}[c]{\ColSource+.7cm}
\setlength{\parindent}{\parindenttemp}%
\par\vspace*{0mm}% pour l'alignement sur le haut
\footnotesize
\begin{verbatim}
\TinyPicture\VCDraw{%
\begin{VCPicture}{(-1,-1)(1,1)}\State[p]{(0,0)}{A}\end{VCPicture}}
\SmallPicture\VCDraw{%
\begin{VCPicture}{(-1,-1)(1,1)}\State[p]{(0,0)}{A}\end{VCPicture}}
\MediumPicture\VCDraw{%
\begin{VCPicture}{(-1,-1)(1,1)}\State[p]{(0,0)}{A}\end{VCPicture}}
\LargePicture\VCDraw{%
\begin{VCPicture}{(-1,-1)(1,1)}\State[p]{(0,0)}{A}\end{VCPicture}}
\end{verbatim}
\normalsize
\end{minipage}% 
\HideFrame

% \medskip
\noi
In the sequel, and unless otherwise stated,
all the figure will be given at the \verb+\MediumPicture+ 
scale without mentionning the command \verb+\VCDraw+ in the source code.

% \smallskipneg 
%%%%%%%%%%%%%%%%%%%%%%
\subsection{State size}\label{subsec.siz1}

There are \emph{three preset sizes} for the states, that are called by the 
commands \verb+\SmallState+, \verb+\MediumState+ and 
\verb+\LargeState+, that call for states of diameter 
\verb+0.6cm+, \verb+0.9cm+ and \verb+1.2cm+ respectively.
The implicit size, \ie the one that states have if no explicit command 
is given, is \verb+\MediumState+.
The ``small'' states are not likely to receive labels, unless their 
size are changed by other commands.
\index{SmallState}\index{MediumState}\index{LargeState}

\noi 
\hspace*{-\jsIndent}
\begin{minipage}[c]{\ColFigur-.7cm}%
\par\vspace*{0mm}% pour l'alignement sur le haut
\begin{center}
\scalebox{\VCScale}{%
\begin{VCPicture}{(0,-2)(6,2)}
% states
\State[A]{(0,0)}{A} 
\SmallState\State{(2,-2)}{B} 
\MediumState\State[C]{(4,2)}{C}
\LargeState\State[D]{(6,0)}{D}
%
\end{VCPicture}%
}%
\end{center}
\end{minipage}%
\hspace*{1.2em}% 
\begin{minipage}[c]{\ColSource+.7cm}
\setlength{\parindent}{\parindenttemp}%
\par\vspace*{0mm}% pour l'alignement sur le haut
\footnotesize
\begin{verbatim}
\begin{VCPicture}{(0,-2)(6,2)}
% states
\State[A]{(0,0)}{A} 
\SmallState\State{(2,-2)}{B} 
\MediumState\State[C]{(4,2)}{C}
\LargeState\State[D]{(6,0)}{D}
%
\end{VCPicture}
\end{verbatim}
\normalsize
\end{minipage}%

% \medskip
\noi
When called outside the scope of the environment \verb+\VCPicture+ the 
commands  \verb+\SmallState+, \verb+\MediumState+ and 
\verb+\LargeState+ set the implicit size of the states in all 
pictures from that point on.
\ShowFrame%

% \smallskipneg 
\noi 
\hspace*{-\jsIndent}
\begin{minipage}[c]{\ColFigur-.7cm}%
\par\vspace*{0mm}% pour l'alignement sur le haut
\begin{center}
        \SmallState
\scalebox{\VCScale}{%
\begin{VCPicture}{(-1,-1)(1,1)}
\State[A]{(0,0)}{A} 
\end{VCPicture}%
}\e 
\LargeState
\scalebox{\VCScale}{%
\begin{VCPicture}{(-1,-1)(1,1)}
\State[A]{(0,0)}{A} 
\end{VCPicture}%
}\e 
\MediumState
\scalebox{\VCScale}{%
\begin{VCPicture}{(-1,-1)(1,1)}
\State[A]{(0,0)}{A} 
\end{VCPicture}%
}
\end{center}
\end{minipage}%
\hspace*{1.2em}% 
\begin{minipage}[c]{\ColSource+.7cm}
\setlength{\parindent}{\parindenttemp}%
\par\vspace*{0mm}% pour l'alignement sur le haut
\footnotesize
\begin{verbatim}
\SmallState
\begin{VCPicture}{(-1,-1)(1,1)}\State[A]{(0,0)}{A} \end{VCPicture}
\LargeState
\begin{VCPicture}{(-1,-1)(1,1)}\State[A]{(0,0)}{A} \end{VCPicture}
\MediumState
\begin{VCPicture}{(-1,-1)(1,1)}\State[A]{(0,0)}{A} \end{VCPicture}
\end{verbatim}
\normalsize
\end{minipage}%
\HideFrame

% \smallskipneg 
\subsection{Variable size states}\label{subsec.varst}
% \medskip
\noi
%Independently of the preset size --- in this instance, of the \emph{height} --- 
One can draw states which will be called \emph{variable states} and whose 
\emph{width} is variable and \emph{fits the width of the 
label}.\footnote{%
   This makes sense for long labels and it may be the case then that 
   the label is not a mathematical variable or expression any more 
   but simple text.
   In order to write labels in \TeX\ LR mode, the easiest way is to 
   write it as the parameter of a $\backslash$\texttt{text} command, 
   provided the \texttt{amsmath} package has been called in the 
   preamble.}
The height of
\index{StateVar}
\index{text}
\index{amsmath}
these variable states is fixed by the current preset state size.

\hspace*{-\jsIndent}
\begin{minipage}[t]{\ColoText}
        \par\vspace*{0mm}% pour l'alignement sur le haut
        \footnotesize
\verb+\StateVar[+\textsl{Label}\verb+]{+%
   $(x,y)$\verb+}{+%
   \textsl{Name}\verb+}+
\normalsize
\end{minipage}% 
\hspace*{1.2em}% 
\begin{minipage}[t]{\ColoFigu}%
% \setlength{\parindent}{\parindenttemp}%
\par\vspace*{0mm}% pour l'alignement sur le haut
Same as syntax as \verb+\State+.
\end{minipage}%
% \clearpage

% \medskip

\noi
Note that the current version of \VCSG uses a trick to make these 
states with adaptable width.
They look like Egyptian cartouches but they are indeed 
\emph{rectangles} which can be observed when transitions reach these 
states in an oblique way.

Arcs between variable states of different width will look asymmetric 
and often ungainly.
Other loops than ``north'' and ``south'' loops should not be drawn 
around variable states. Moreover, in order to draw loops that have the same appearance as loops
around standard states, one should use 
\verb+\LoopVarN+ or \verb+\LoopVarS+
that have the same syntax as \verb+\LoopN+ or \verb+\LoopS+.
\index{LoopVarN}\index{LoopVarS}
Another possibility is to call the \verb+\VarLoopOn+ command to make 
all the subsequent loops suited to variable states.
Switching back to `normal' loops is made by \verb+\VarLoopOff+.
\index{VarLoopOn}
\index{VarLoopOff}

For the same reason, it is \emph{not possible} to make a variable
state initial or final with diagonal arrows ($ne$, $nw$, $se$, $sw$).


\medskip 

\noi 
\hspace*{-\jsIndent}
\begin{minipage}[c]{\ColFigur}%
\par\vspace*{0mm}% pour l'alignement sur le haut
\begin{center}
\SmallPicture\VCDraw{%
\begin{VCPicture}{(0,-3)(8,3)}
% states
\State[AA]{(0,2)}{A} \StateVar[AA]{(2.5,2)}{B} 
\StateVar[(b a^{*} b)^{*}]{(7,2)}{C} 
\LargeState 
\State[AA]{(0,-2)}{A1} \StateVar[AA]{(2.5,-2)}{B1} 
\StateVar[(b a^{*} b)^{*}]{(7,-2)}{C1} 
%
\Initial[n]{B} \Final{C} \Initial[s]{B1} \Final{C1}
%
\EdgeL{A}{B}{a} \ArcL{B}{C}{b} \ArcL{C}{C1}{c} \EdgeL{B1}{C}{a} 
\ArcL{A1}{B1}{a} \LArcR{A1}{B1}{b} \LArcR{B1}{C1}{b} 
\LoopN{A}{c} \LoopVarN{C}{c}
\LoopS{A1}{c} \VarLoopOn\LoopS{C1}{c}\VarLoopOff
\end{VCPicture}}%
\end{center}
\end{minipage}%
\hspace*{1.2em}% 
\begin{minipage}[c]{\ColSource}
\setlength{\parindent}{\parindenttemp}%
\par\vspace*{0mm}% pour l'alignement sur le haut
\footnotesize
\begin{verbatim}
\SmallPicture\VCDraw{\begin{VCPicture}{(0,-3)(8,3)}
% states
\State[AA]{(0,2)}{A} \StateVar[AA]{(2.5,2)}{B} 
\StateVar[(b a^{*} b)^{*}]{(7,2)}{C} 
\LargeState \StateVar[(b a^{*} b)^{*}]{(7,-2)}{C1} 
\State[AA]{(0,-2)}{A1} \StateVar[AA]{(2.5,-2)}{B1} 
%
\Initial[n]{B} \Final{C} \Initial[s]{B1} \Final{C1}
%
\EdgeL{A}{B}{a} \ArcL{B}{C}{b} \ArcL{C}{C1}{c} \EdgeL{B1}{C}{a} 
\ArcL{A1}{B1}{a} \LArcR{A1}{B1}{b} \LArcR{B1}{C1}{b} 
\LoopN{A}{c} \LoopVarN{C}{c}
\LoopS{A1}{c} \VarLoopOn\LoopS{C1}{c}\VarLoopOff
\end{VCPicture}}
\end{verbatim}
\normalsize
\end{minipage}

\medskip 

\noi
There exists\footnote{%
   Since version 0.3 only.}
also a \verb+\FinalStateVar+  command, with the same syntax as 
\verb+\StateVar+ and which draws a variable state with double line, 
(independently of the current drawing style for the state), exactly 
in the same way as does \verb+\FinalState+.
\index{FinalStateVar}%

\subsection{Very small states}\label{subsec.siz2}
There is actually a fourth size for states, called \emph{very small} states.
To define such a state one uses the command \verb+\VSState+ with
the same syntax as \verb+\State+. However, it is not possible, of course,
to put a label inside such a state.
\index{VSState}

\medskip 

\noi 
\hspace*{-\jsIndent}
\begin{minipage}[c]{\ColFigur}%
\par\vspace*{0mm}% pour l'alignement sur le haut
\begin{center}
\SmallPicture\VCDraw{%
\begin{VCPicture}{(-3,-3)(3,3)}
% states
\State[A]{(0,0)}{A}
\VSState{(-2,2)}{G1} 
\VSState{(-2,0)}{G2} 
\VSState{(-2,-2)}{G3} 
\VSState{(2,2)}{D1} 
\VSState{(2,0)}{D2} 
\VSState{(2,-2)}{D3} 
\EdgeL{G1}{A}{a_1} \EdgeL{G2}{A}{a_2} \EdgeL{G3}{A}{a_3}
\EdgeL{A}{D1}{b_1} \EdgeL{A}{D2}{b_2} \EdgeL{A}{D3}{b_3}
\end{VCPicture}}%
\end{center}
\end{minipage}%
\hspace*{1.2em}% 
\begin{minipage}[c]{\ColSource}
\setlength{\parindent}{\parindenttemp}%
\par\vspace*{0mm}% pour l'alignement sur le haut
\footnotesize
\begin{verbatim}
\begin{VCPicture}{(-3,-3)(3,3)}
% states
\State[A]{(0,0)}{A}
\VSState{(-2,2)}{G1} \VSState{(-2,0)}{G2} \VSState{(-2,-2)}{G3}
\VSState{(2,2)}{D1} \VSState{(2,0)}{D2} \VSState{(2,-2)}{D3}
\EdgeL{G1}{A}{a_1} \EdgeL{G2}{A}{a_2} \EdgeL{G3}{A}{a_3}
\EdgeL{A}{D1}{b_1} \EdgeL{A}{D2}{b_2} \EdgeL{A}{D3}{b_3}
\end{VCPicture}}
\end{verbatim}
\normalsize
\end{minipage}%

%%%%%%%%%%%%%%%%%%%%%%
\subsection{Miscellaneous}\label{subsec.misc1}

A few commands allow to tune the drawing of an automaton without 
going to precise mastering of the parameters involved in the commands 
and that we shall describe in section~4.

%%%%%%%%%%%%%%%%%%%%%%
\paragraph{Dimmed states and transitions}
~

\noi 
\hspace*{-\jsIndent}
\begin{minipage}[t]{\ColoText}
        \par\vspace*{0mm}% pour l'alignement sur le haut
        \footnotesize
\verb+\DimState+ \ee 
\verb+\DimEdge+
   
\medskip 
\verb+\RstState+ \ee 
\verb+\RstEdge+
   
\medskip 
\end{minipage}% 
\hspace*{1.2em}% 
\begin{minipage}[t]{\ColoFigu}%
% \setlength{\parindent}{\parindenttemp}%
\par\vspace*{0mm}% pour l'alignement sur le haut
Draw states and transitions and their 
respective labels in gray.

% , in such a way they appear as 
% dimmed. 
\smallskip 

Restore the normal style.
\end{minipage}%
\index{DimState}\index{DimEdge}
\index{RstEdge}\index{RstState}

\medskip
\noi 
May be used for instance in order to indicate the non accessible part 
of an automaton.

\noi 
\hspace*{-\jsIndent}
\begin{minipage}[c]{\ColFigur}%
\par\vspace*{0mm}% pour l'alignement sur le haut
\begin{center}
\scalebox{\VCScale}{%
\begin{VCPicture}{(0,-2)(6,2)}
%% states
\State[A]{(0,0)}{A} \State[B]{(2,-2)}{B} \State[C]{(4,2)}{C}
\DimState \State[D]{(6,0)}{D}
%initial-final
\Initial{A} \Final[s]{B} \Final[n]{C}
% transitions 
\ArcR{A}{B}{a} \ArcL{A}{C}{b} 
\DimEdge \LArcR{B}{D}{c} \LArcL{C}{D}{d} \RstEdge
\ArcL{A}{B}{a'} \EdgeR{A}{C}{b'}
\DimEdge \EdgeL{B}{D}{c'} \LArcR{C}{D}{d'}
%
\end{VCPicture}%
}%
\end{center}
\end{minipage}%
\hspace*{1.2em}% 
\begin{minipage}[c]{\ColSource}
\setlength{\parindent}{\parindenttemp}%
\par\vspace*{0mm}% pour l'alignement sur le haut
\footnotesize
\begin{verbatim}
\begin{VCPicture}{(0,-2)(6,2)}
% states
\State[A]{(0,0)}{A} \State[B]{(2,-2)}{B} \State[C]{(4,2)}{C}
\DimState \State[D]{(6,0)}{D}
%initial-final
\Initial{A} \Final[s]{B} \Final[n]{C}
% transitions 
\ArcR{A}{B}{a} \ArcL{A}{C}{b} 
\DimEdge \LArcR{B}{D}{c} \LArcL{C}{D}{d} \RstEdge
\ArcL{A}{B}{a'} \EdgeR{A}{C}{b'}
\DimEdge \EdgeL{B}{D}{c'} \LArcR{C}{D}{d'}
%
\end{VCPicture}
\end{verbatim}
\normalsize
\end{minipage}% 

%%%%%%%%%%%%%%%%%%%%%%
\paragraph{Double lines}
~

\noi 
\hspace*{-\jsIndent}
\begin{minipage}[t]{\ColoText}
        \par\vspace*{0mm}% pour l'alignement sur le haut
        \footnotesize
\verb+\StateLineDouble+ \ee \verb+\EdgeLineDouble+ 
   
\medskip 
\verb+\StateLineSimple+ \ee \verb+\EdgeLineSimple+ 
   
\end{minipage}% 
\hspace*{1.2em}% 
\begin{minipage}[t]{\ColoFigu}%
% \setlength{\parindent}{\parindenttemp}%
\par\vspace*{0mm}% pour l'alignement sur le haut
Draw states and transitions with double lines.
\index{StateLineDouble}\index{EdgeLineDouble}

\smallskip 

Draw states and transitions with simple lines.
\index{StateLineSimple}\index{EdgeLineSimple}
\end{minipage}%

\medskip
\noi 
May be used for emphasize transitions or states 
of an automaton.

\noi 
\hspace*{-\jsIndent}
\begin{minipage}[c]{\ColFigur}%
\par\vspace*{0mm}% pour l'alignement sur le haut
\begin{center}
\scalebox{\VCScale}{%
\begin{VCPicture}{(0,-2)(6,2)}
% states
\State[A]{(0,0)}{A} \State[B]{(2,-2)}{B}
\StateLineDouble \State[C]{(4,2)}{C}
\StateLineSimple \State[D]{(6,0)}{D}
%initial-final
\Initial{A} \Final[s]{B}
% transitions 
\ArcR{A}{B}{a} \ArcL{A}{C}{b}
\LArcR{B}{D}{c} \LArcL{C}{D}{d}
\ArcL{A}{B}{a'} \EdgeLineDouble \EdgeR{A}{C}{b'}
\EdgeLineSimple \EdgeL{B}{D}{c'} \LArcR{C}{D}{d'}
%
\end{VCPicture}%
}%
\end{center}
\end{minipage}%
\hspace*{1.2em}% 
\begin{minipage}[c]{\ColSource}
\setlength{\parindent}{\parindenttemp}%
\par\vspace*{0mm}% pour l'alignement sur le haut
\footnotesize
\begin{verbatim}
\begin{VCPicture}{(0,-2)(6,2)}
% states
\State[A]{(0,0)}{A} \State[B]{(2,-2)}{B}
\StateLineDouble \State[C]{(4,2)}{C}
\StateLineSimple \State[D]{(6,0)}{D}
%initial-final
\Initial{A} \Final[s]{B}
% transitions 
\ArcR{A}{B}{a} \ArcL{A}{C}{b}
\LArcR{B}{D}{c} \LArcL{C}{D}{d}
\ArcL{A}{B}{a'} \EdgeLineDouble \EdgeR{A}{C}{b'}
\EdgeLineSimple \EdgeL{B}{D}{c'} \LArcR{C}{D}{d'}
%
\end{VCPicture}
\end{verbatim}
\normalsize
\end{minipage}% 

\noi
The command \verb+\FinalState{+$(x,y)$\verb+}{+$p$\verb+}+ is equivalent to
the sequence\\
\verb+\StateLineDouble \State{+$(x,y)$\verb+}{+$p$\verb+} \StateLineSimple+ 
if the current implicit mode is \verb+\StateLineSimple+ , 
it is equivalent to \verb+\State{+$(x,y)$\verb+}{+$p$\verb+}+ in the 
other case.


%%%%%%%%%%%%%%%%%%%%%%
\paragraph{Hiding states}
~

\noi 
\hspace*{-\jsIndent}
\begin{minipage}[t]{\ColoText}
        \par\vspace*{0mm}% pour l'alignement sur le haut
        \footnotesize
\verb+\HideState+ \ee 
   
\medskip 
\verb+\ShowState+ \ee 
   
\end{minipage}% 
\hspace*{1.2em}% 
\begin{minipage}[t]{\ColoFigu}%
% \setlength{\parindent}{\parindenttemp}%
\par\vspace*{0mm}% pour l'alignement sur le haut
Cancel the drawing of states.\index{HideState}

\smallskip 

Active the drawing of states.\index{ShowState}
\end{minipage}%

\medskip
\noi 
May be useful for drawing slides that will be overlayed.\footnote{%
   We began to write the package when overhead projectors were still 
   of common usage\ldots}

\noi 
\hspace*{-\jsIndent}
\begin{minipage}[c]{\ColFigur}%
\par\vspace*{0mm}% pour l'alignement sur le haut
\begin{center}
\scalebox{\VCScale}{%
\begin{VCPicture}{(0,-2)(6,2)}
% states
\HideState
\State[i]{(0,0)}{A} \State[t]{(6,0)}{C}
\ShowState
\State[q]{(3,0)}{B} 
% initial-final
\Initial{A} \Final{C}
% transitions 
\EdgeR{A}{B}{2b} \EdgeR{B}{C}{2b} \LArcL{A}{C}{b}
\LoopN{A}{a} \LoopS{A}{b} \LoopS[.5]{B}{2a+2b} 
\LoopN[.75]{C}{4a} \LoopS[.75]{C}{4b}
%
\end{VCPicture}%
}%
\end{center}
\end{minipage}%
\hspace*{1.2em}% 
\begin{minipage}[c]{\ColSource}
\setlength{\parindent}{\parindenttemp}%
\par\vspace*{0mm}% pour l'alignement sur le haut
\footnotesize
\begin{verbatim}
\begin{VCPicture}{(0,-2)(6,2)}
% states
\HideState
\State[i]{(0,0)}{A} \State[t]{(6,0)}{C}
\ShowState
\State[q]{(3,0)}{B} 
% initial-final
\Initial{A} \Final{C}
% transitions 
\EdgeR{A}{B}{2b} \EdgeR{B}{C}{2b} \LArcL{A}{C}{b}
\LoopN{A}{a} \LoopS{A}{b} \LoopS[.5]{B}{2a+2b} 
\LoopN[.75]{C}{4a} \LoopS[.75]{C}{4b}
%
\end{VCPicture}
\end{verbatim}
\normalsize
\end{minipage}% 

%%%%%%%%%%%%%%%%%%%%%%
\paragraph{Reverse transitions}
~

\noi 
\hspace*{-\jsIndent}
\begin{minipage}[t]{\ColoText}
        \par\vspace*{0mm}% pour l'alignement sur le haut
        \footnotesize
\verb+\ReverseArrow+ \ee 
   
\medskip 
\verb+\StraightArrow+ \ee 
   
\end{minipage}% 
\hspace*{1.2em}% 
\begin{minipage}[t]{\ColoFigu}%
% \setlength{\parindent}{\parindenttemp}%
\par\vspace*{0mm}% pour l'alignement sur le haut
Exchange the origin and the end of a transition.\index{ReverseArrow}

\smallskip 

Restore the normal style.\index{StraightArrow}
\end{minipage}%

\medskip
\noi 
May be useful for modifying a part of a figure that had already been 
written.

\noi 
\hspace*{-\jsIndent}
\begin{minipage}[c]{\ColFigur}%
\par\vspace*{0mm}% pour l'alignement sur le haut
\bigskip 
\begin{center}
\scalebox{\VCScale}{%
\begin{VCPicture}{(0,-2)(6,2)}
% states
\State[A]{(0,0)}{A} \State[B]{(2,-2)}{B} \State[C]{(4,2)}{C}
\State[D]{(6,0)}{D}
% transitions 
\Final[s]{B}
\ArcR{A}{B}{a} \ArcL{A}{C}{b} \LArcR{B}{D}{c} \LArcL{C}{D}{d}
\ReverseArrow
\Initial{A} \Final[n]{C}
\ArcL{A}{B}{a'} \EdgeR{A}{C}{b'} \EdgeL{B}{D}{c'} \LArcR{C}{D}{d'}
%
\end{VCPicture}%
}%
\end{center}
\end{minipage}%
\hspace*{1.2em}% 
\begin{minipage}[c]{\ColSource}
\setlength{\parindent}{\parindenttemp}%
\par\vspace*{0mm}% pour l'alignement sur le haut
\footnotesize
\begin{verbatim}
\begin{VCPicture}{(0,-2)(6,2)}
% states
\State[A]{(0,0)}{A} \State[B]{(2,-2)}{B} \State[C]{(4,2)}{C}
\State[D]{(6,0)}{D} 
% transitions 
\Final[s]{B}
\ArcR{A}{B}{a} \ArcL{A}{C}{b} \LArcR{B}{D}{c} \LArcL{C}{D}{d}
\ReverseArrow
\Initial{A} \Final[n]{C}
\ArcL{A}{B}{a'} \EdgeR{A}{C}{b'} \EdgeL{B}{D}{c'}
\LArcR{C}{D}{d'}
%
\end{VCPicture}
\end{verbatim}
\normalsize
\end{minipage}% 

%%%%%%%%%%%%%%%%%%%%%%
\paragraph{Edge border}
~

\noi 
\hspace*{-\jsIndent}
\begin{minipage}[t]{\ColoText}
        \par\vspace*{0mm}% pour l'alignement sur le haut
        \footnotesize
\verb+\EdgeBorder+ \ee 
   
\medskip 
\verb+\EdgeBorderOff+ \ee 
   
\end{minipage}% 
\hspace*{1.2em}% 
\begin{minipage}[t]{\ColoFigu}%
% \setlength{\parindent}{\parindenttemp}%
\par\vspace*{0mm}% pour l'alignement sur le haut
Draw edges with a white border on each side.\index{EdgeBorder}

\smallskip 

Restore the normal style.\index{EdgeBorderOff}
\end{minipage}%

\medskip
\noi
May be used to make more readable a picture in which many arrows 
cross each other.
The \emph{order} in which the edges are drawn becomes then meaningfull.

\noi 
\hspace*{-\jsIndent}
\begin{minipage}[c]{\ColFigur}%
\par\vspace*{0mm}% pour l'alignement sur le haut
\begin{center}
\scalebox{\VCScale}{%
\begin{VCPicture}{(0,-2)(6,2)}
% states
\State[A]{(0,1)}{A}  \State[B]{(3,2)}{B}  \State[C]{(6,1)}{C}
\State[D]{(0,-1)}{D} \State[E]{(3,-2)}{E} \State[F]{(6,-1)}{F}
% transitions 
\EdgeR{A}{E}{a} \EdgeL[.8]{D}{B}{b} \EdgeR[.7]{B}{E}{b} 
%
\EdgeBorder
\EdgeL[.35]{A}{F}{a} \EdgeR[.3]{E}{C}{d}
\EdgeBorderOff
\EdgeL{B}{F}{c}
\end{VCPicture}%
}%
\end{center}
\end{minipage}%
\hspace*{1.2em}% 
\begin{minipage}[c]{\ColSource}
\setlength{\parindent}{\parindenttemp}%
\par\vspace*{0mm}% pour l'alignement sur le haut
\footnotesize
\begin{verbatim}
\begin{VCPicture}{(0,-2)(6,2)}
% states
\State[A]{(0,1)}{A}  \State[B]{(3,2)}{B}  \State[C]{(6,1)}{C}
\State[D]{(0,-1)}{D} \State[E]{(3,-2)}{E} \State[F]{(6,-1)}{F}
% transitions 
\EdgeR{A}{E}{a} \EdgeL[.8]{D}{B}{b} \EdgeR[.7]{B}{E}{b} 
%
\EdgeBorder
\EdgeL[.35]{A}{F}{a} \EdgeR[.3]{E}{C}{d}
\EdgeBorderOff
\EdgeL{B}{F}{c}
\end{VCPicture}
\end{verbatim}
\normalsize
\end{minipage}% 


%%%%%%%%%%%%%%%%%%%%%%
\paragraph{Forth and back offset}
~
Normally, an edge lays on the line that links the center of the start 
and end states.
Which forbid to have two straight edges in opposite directions 
between states.

\noi 
\hspace*{-\jsIndent}
\begin{minipage}[t]{\ColoText}
        \par\vspace*{0mm}% pour l'alignement sur le haut
        \footnotesize
\verb+\ForthBackOffset+ \ee 
   
\medskip 
\verb+\RstEdgeOffset+ \ee 
   
\end{minipage}% 
\hspace*{1.2em}% 
\begin{minipage}[t]{\ColoFigu}%
% \setlength{\parindent}{\parindenttemp}%
\par\vspace*{0mm}% pour l'alignement sur le haut
Shift the edge on the left handside.\index{ForthBackOffset}

\smallskip 

Restore the normal style.\index{RstEdgeOffset}
\end{minipage}%

\medskip
\noi 
\hspace*{-\jsIndent}
\begin{minipage}[c]{\ColFigur}%
\par\vspace*{0mm}% pour l'alignement sur le haut
\bigskip 
\begin{center}
\scalebox{\VCScale}{%
\begin{VCPicture}{(0,-2)(6,2)}
% states
\State[A]{(0,1)}{A} \State[B]{(3,1)}{B} \State[C]{(6,1)}{C}
\LargeState
\State[A]{(0,-1)}{A1} \State[B]{(3,-1)}{B1} \State[C]{(6,-1)}{C1}
% transitions 
\EdgeL{A}{B}{a} 
\ForthBackOffset \EdgeL{B}{C}{b} \EdgeL{C}{B}{b} 
\RstEdgeOffset 
\EdgeL{A1}{B1}{a} 
\ForthBackOffset \EdgeL{B1}{C1}{b} \EdgeL{C1}{B1}{b} 
%
\end{VCPicture}%
}%
\end{center}
\end{minipage}%
\hspace*{1.2em}% 
\begin{minipage}[c]{\ColSource}
\setlength{\parindent}{\parindenttemp}%
\par\vspace*{0mm}% pour l'alignement sur le haut
\footnotesize
\begin{verbatim}
\begin{VCPicture}{(0,-2)(6,2)}
% states
\State[A]{(0,1)}{A} \State[B]{(3,1)}{B} \State[C]{(6,1)}{C}
\LargeState
\State[A]{(0,-1)}{A1} \State[B]{(3,-1)}{B1}
\State[C]{(6,-1)}{C1}
% transitions 
\EdgeL{A}{B}{a} 
\ForthBackOffset \EdgeL{B}{C}{b} \EdgeL{C}{B}{b} 
\RstEdgeOffset \EdgeL{A1}{B1}{a} 
\ForthBackOffset \EdgeL{B1}{C1}{b} \EdgeL{C1}{B1}{b} 
\end{VCPicture}
\end{verbatim}
\normalsize
\end{minipage}% 

%%%%%%%%%%%%%%%%%%%%%%
\paragraph{Zigzag edges}
~

\noi 
\hspace*{-\jsIndent}
\begin{minipage}[t]{\ColoText}
        \par\vspace*{0mm}% pour l'alignement sur le haut
\medskip 
        \footnotesize
\verb+\ZZEdgeL[+\textsl{Pos}\verb+]{+%
   \textsl{Start}\verb+}{+%
   \textsl{End}\verb+}{+%
   \textsl{Label}\verb+}+
   
\medskip 
\verb+\ZZEdgeR[+\textsl{Pos}\verb+]{+%
   \textsl{Start}\verb+}{+%
   \textsl{End}\verb+}{+%
   \textsl{Label}\verb+}+
   \normalsize
\end{minipage}% 
\hspace*{1.2em}% 
\begin{minipage}[t]{\ColoFigu}%
% \setlength{\parindent}{\parindenttemp}%
\par\vspace*{0mm}% pour l'alignement sur le haut
Draws a ``zigzag'' transition from the state \textsl{Start} to the 
state \textsl{End} with the label \textsl{Label} on the left or
the right side.\index{ZZEdge}
\end{minipage}%

\noi 
\hspace*{-\jsIndent}
\begin{minipage}[c]{\ColFigur}%
\par\vspace*{0mm}% pour l'alignement sur le haut
\begin{center}
\scalebox{\VCScale}{%
\begin{VCPicture}{(0,-2)(6,2)}
% states
\State[A]{(0,0)}{A} \State[B]{(2,-2)}{B} \State[C]{(4,2)}{C}
\VSState{(6,0)}{D}
% transitions 
\EdgeR{A}{B}{a} \EdgeL{A}{C}{b} \EdgeR{B}{D}{c} 
\ZZEdgeL{C}{D}{d} \ZZEdgeR{C}{B}{b}
%
\end{VCPicture}%
}%
\end{center}
\end{minipage}%
\hspace*{1.2em}% 
\begin{minipage}[c]{\ColSource}
\setlength{\parindent}{\parindenttemp}%
\par\vspace*{0mm}% pour l'alignement sur le haut
\footnotesize
\begin{verbatim}
\begin{VCPicture}{(0,-2)(6,2)}
% states
\State[A]{(0,0)}{A} \State[B]{(2,-2)}{B} \State[C]{(4,2)}{C}
\VSState{(6,0)}{D}
% transitions 
\EdgeR{A}{B}{a} \EdgeL{A}{C}{b} \EdgeR{B}{D}{c} 
\ZZEdgeL{C}{D}{d} \ZZEdgeR{C}{B}{b}
%
\end{VCPicture}
\end{verbatim}
\normalsize
\end{minipage}% 

\paragraph{Lining up}
~
By default, labels of states are centered. When states are aligned, it may happen
that this seems ungainly. The following command solves this problem.

\noi 
\hspace*{-\jsIndent}
\begin{minipage}[t]{\ColoText}
        \par\vspace*{0mm}% pour l'alignement sur le haut
        \footnotesize
\verb+\AlignedLabel+ \ee 
   
\medskip 
\verb+\FloatingLabel+ \ee 
   
\end{minipage}% 
\hspace*{1.2em}% 
\begin{minipage}[t]{\ColoFigu}%
% \setlength{\parindent}{\parindenttemp}%
\par\vspace*{0mm}% pour l'alignement sur le haut
Put labels of states on a virtual baseline.\index{AlignedLabel}

\smallskip 

Center the label of states vertically.\index{FloatingLabel}
\end{minipage}%

\noi
Default option is \verb+\FloatingLabel+.

\noi 
\hspace*{-\jsIndent}
\begin{minipage}[c]{\ColFigur}%
\par\vspace*{0mm}% pour l'alignement sur le haut
\bigskip 
\begin{center}
\scalebox{\VCScale}{%
\begin{VCPicture}{(0,-4)(6,1)}
% states
\State[p]{(0,0)}{A} \State[b]{(3,0)}{B} \State[a]{(6,0)}{C}
\AlignedLabel
\State[p]{(0,-3)}{A1} \State[b]{(3,-3)}{B1} \State[a]{(6,-3)}{C1}
%transitions 
\EdgeL{A}{B}{a} \ArcL{B}{C}{b} \ArcL{C}{B}{c} 
\EdgeL{A1}{B1}{a} \ArcL{B1}{C1}{b} \ArcL{C1}{B1}{c} 
%
\end{VCPicture}%
}%
\end{center}
\end{minipage}%
\hspace*{1.2em}% 
\begin{minipage}[c]{\ColSource}
\setlength{\parindent}{\parindenttemp}%
\par\vspace*{0mm}% pour l'alignement sur le haut
\footnotesize
\begin{verbatim}
\begin{VCPicture}{(0,-4)(6,1)}
% states
\State[p]{(0,0)}{A} \State[b]{(3,0)}{B} \State[a]{(6,0)}{C}
\AlignedLabel
\State[p]{(0,-3)}{A1} \State[b]{(3,-3)}{B1}
\State[a]{(6,-3)}{C1}
%transitions 
\EdgeL{A}{B}{a} \ArcL{B}{C}{b} \ArcL{C}{B}{c} 
\EdgeL{A1}{B1}{a} \ArcL{B1}{C1}{b} \ArcL{C1}{B1}{c} 
%
\end{VCPicture}%
\end{verbatim}
\normalsize
\end{minipage}% 



%%%%%%%%%%%%%%%%%%%%%%
\paragraph{The command \texttt{$\backslash$Point}}
~
This commands is a direct translation of a command from \PSTricks.

\noi 
\hspace*{-\jsIndent}
\begin{minipage}[t]{\ColoText}
        \par\vspace*{0mm}% pour l'alignement sur le haut
        \footnotesize
\verb+\Point{+%
   $(x,y)$\verb+}{+%
   \textsl{Name}\verb+}+
   
\end{minipage}% 
\hspace*{1.2em}% 
\begin{minipage}[t]{\ColoFigu}%
% \setlength{\parindent}{\parindenttemp}%
\par\vspace*{0mm}% pour l'alignement sur le haut
Sets a point with name \textsl{Name} at the coordinate $(x,y)$.\index{Point}
\end{minipage}% 

This command does \textsl{not} draw any point.
With \PSTricks, and thus with \VCSG, any line or curve must be drawn
\textsl{between} two predefined objects. That is the aim of \verb+\Point+.
For instance, initial and final arrows are drawn with this trick:
any time a state is defined, eight points are silently and implicitely defined around it.

%\noi 
%\hspace*{-\jsIndent}
%\begin{minipage}[t]{\ColoText}
%       \par\vspace*{0mm}% pour l'alignement sur le haut
%       \footnotesize
%\verb+\Edge{+%
%   \textsl{Start}\verb+}{+%
%   \textsl{End}\verb+}+
%   
%\end{minipage}% 
%\hspace*{1.2em}% 
%\begin{minipage}[t]{\ColoFigu}%
%% \setlength{\parindent}{\parindenttemp}%
%\par\vspace*{0mm}% pour l'alignement sur le haut
%Sets a point with name \textsl{Name} at the coordinate \textsl{x,y}.
%
%\smallskip 
%
%Draws a straight arrow from the state \textsl{Start} to the state \textsl{End}.
%\end{minipage}%
%
%\medskip

\subsection{Writing composite labels}
Some useful tools are proposed to write composite labels on arrows.

\noi 
\hspace*{-\jsIndent}
\begin{minipage}[t]{\ColoText}
        \par\vspace*{0mm}% pour l'alignement sur le haut
        \footnotesize
\verb+\IOL{+%
   \textsl{Input}\verb+}{+%
   \textsl{Output}\verb+}+
\end{minipage}% 
\hspace*{1.2em}% 
\begin{minipage}[t]{\ColoFigu}%
% \setlength{\parindent}{\parindenttemp}%
\par\vspace*{0mm}% pour l'alignement sur le haut
Produces the string \textsl{Input}$\mid$\textsl{Output}, useful
when one draws transducers. This allows to modify the way it is done
by redefining command \verb+\IOL+.\index{IOL}

\end{minipage}%

\noi 
\hspace*{-\jsIndent}
\begin{minipage}[t]{\ColoText}
        \par\vspace*{0mm}% pour l'alignement sur le haut
        \footnotesize
\verb+\Stack+\texttt{\textsl{X}}\verb+Labels{+%
   \textsl{Label1}\verb+}{+%
   \textsl{Label2}\verb+}+

\medskip
\verb+\Stack+\texttt{\textsl{X}}\verb+LabelsP{+%
   \textsl{Label1}\verb+}{+%
   \textsl{Label2}\verb+}+

\medskip
\verb+\Line+\texttt{\textsl{X}}\verb+LabelsP{+%
   \textsl{Label1}\verb+}{+%
   \textsl{Label2}\verb+}+

\end{minipage}% 
\hspace*{1.2em}% 
\begin{minipage}[t]{\ColoFigu}%
% \setlength{\parindent}{\parindenttemp}%
\par\vspace*{0mm}% pour l'alignement sur le haut
In these commands, \texttt{\textsl{X}} equals \texttt{Two} or \texttt{Three}.
In the later case, commands have obviously three arguments.

\verb+\Stack+\texttt{\textsl{X}}\verb+Labels+ gives a vertical arrangement of labels, whereas
\verb+\Stack+\texttt{\textsl{X}}\verb+LabelsP+ inserts symbols $+$ between them.
\verb+\Line+\texttt{\textsl{X}}\verb+LabelsP+ is the horizontal version of \verb+\Stack+\texttt{\textsl{X}}\verb+LabelsP+;
it can be redefined in order to change symbols, for instance.
\index{StackLabels}\index{LineLabelsP}
\end{minipage}%

\noi
The following automaton is the right sequential transducer that realizes the addition in base~$2$.

\noi 
\hspace*{-\jsIndent}
\begin{minipage}[c]{\ColFigur}%
\par\vspace*{0mm}% pour l'alignement sur le haut
\begin{center}
\scalebox{\VCScale}{%
\begin{VCPicture}{(-2,-2)(5,3)}
% states
\State[0]{(0,0)}{A} \State[1]{(3,0)}{B}
\Initial[n]{A}
\Final[s]{A} \FinalR{e}{B}{\IOL{}{1}}
%transitions 
\LoopW[.5]{A}{\StackTwoLabelsP{\IOL{0}{0}}{\IOL{1}{1}}}
\LoopN[.7]{B}{\LineTwoLabelsP{\IOL{1}{0}}{\IOL{2}{1}}}
\ArcL{A}{B}{\IOL{2}{0}}
\ArcL{B}{A}{\IOL{0}{1}}
%
\end{VCPicture}%
}%
\end{center}
\end{minipage}%
\hspace*{1.2em}% 
\begin{minipage}[c]{\ColSource}
\setlength{\parindent}{\parindenttemp}%
\par\vspace*{0mm}% pour l'alignement sur le haut
\footnotesize
\begin{verbatim}
\begin{VCPicture}{(-2,-2)(5,3)}
% states
\State[0]{(0,0)}{A} \State[1]{(3,0)}{B}
\Initial[n]{A}
\Final[s]{A} \FinalR{e}{B}{\IOL{}{1}}
%transitions 
\LoopW[.5]{A}{\StackTwoLabelsP{\IOL{0}{0}}{\IOL{1}{1}}}
\LoopN[.7]{B}{\LineTwoLabelsP{\IOL{1}{0}}{\IOL{2}{1}}}
\ArcL{A}{B}{\IOL{2}{0}}
\ArcL{B}{A}{\IOL{0}{1}}
%
\end{VCPicture}
\end{verbatim}
\normalsize
\end{minipage}% 

%%%%%%%%%%%%%%%%%%%%%%
\section{Composing a picture}\label{sec.com}

The commands described above allow to define most of the figures 
representing automata in any textbook or technical paper.
Before turning to more sophisticated commands that are used 
for less than 5 percent of the elements and that appear in less than 
10 percent of the figures, let us turn to few commands that help in 
composing the figures.


%%%%%%%%%%%%%%%%%%%%%%
\subsection{Structuring the automaton: The \texttt{VCPut} command}\label{subsec.rpu}

In many automata, a same pattern in states, and also in edges, is 
reproduced several times in the figure.
The idea is not to try to ``program'' these regular patterns but to 
allow the use of the ``copy-and-paste'' operations in the source 
file as much as possible, with minimum corrections afterwards.
This will be achieved easily with the help of the  
command~\verb+\VCPut+.\index{VCPut}


\noi 
\hspace*{-\jsIndent}
\begin{minipage}[t]{\ColoText}
        \par\vspace*{0mm}% pour l'alignement sur le haut
        \footnotesize
\verb+\VCPut{(+$x$,$y$\verb+)}{+ instructions\ldots \verb+}+
\normalsize
\end{minipage}% 
\hspace*{1.2em}% 
\begin{minipage}[t]{\ColoFigu}%
% \setlength{\parindent}{\parindenttemp}%
\par\vspace*{0mm}% pour l'alignement sur le haut
Translates the result of the instructions by the vector~($x$, $y$).
\end{minipage}%

\medskip
\noi
Note that this command is taken directly from \PSTricks but uses \VCSG's syntax.
%the special syntax of that command which is taken directly from 
%PSTricks: the first argument ($x$,$y$)
%\emph{is not} put inside braces.
%There should not be any space before, after, and inside the 
%parentheses.

Of course the commands that are to be put as the second argument 
of~\verb+\VCPut+ are the defining states commands, with possibly  
other~\verb+\VCPut+ inside. 
The structuration of the figure \textit{via} the use of~\verb+\VCPut+
commands does not only save typing.
It makes also easier the tuning of the figure, and the reusability of 
its parts.

In the example below, the two~\verb+\VCPut+ are used to place the 
two parts of the automaton.

\medskip
\noi 
\hspace*{-\jsIndent}
\begin{minipage}[c]{\ColFigur}%
\par\vspace*{0mm}% pour l'alignement sur le haut
\begin{center}
%       \ShowFrame 
\SmallPicture\VCDraw{%
\begin{VCPicture}{(-1,-2)(7,5)}
\VCPut{(0,3)}{\State[j]{(0,0)}{A1} \State[r]{(3,0)}{B1}}
\VCPut{(3,0)}{\State[s]{(0,0)}{A2} \State[u]{(3,0)}{B2}}
%
\Initial{A1} \Final{B2} 
\EdgeL{A1}{B1}{b} \LoopN[.5]{A1}{a+b} \LoopN[.5]{B1}{2a+2b}
\EdgeR{A2}{B2}{b} \LoopS[.5]{A2}{2a+2b} \LoopS[.5]{B2}{4a+4b}
\EdgeR{A1}{A2}{b} \EdgeL{B1}{B2}{b} \EdgeL{A1}{B2}{b}
%
\end{VCPicture}
}%
\end{center}
\end{minipage}%
\hspace*{1.2em}% 
\begin{minipage}[c]{\ColSource}
\setlength{\parindent}{\parindenttemp}%
\par\vspace*{0mm}% pour l'alignement sur le haut
\footnotesize
\begin{verbatim}
\begin{VCPicture}{(-1,-2)(7,5)}
\VCPut{(0,3)}{\State[j]{(0,0)}{A1} \State[r]{(3,0)}{B1}}
\VCPut{(3,0)}{\State[s]{(0,0)}{A2} \State[u]{(3,0)}{B2}}
%
\Initial{A1} \Final{B2}
\EdgeL{A1}{B1}{b} \LoopN[.5]{A1}{a+b} \LoopN[.5]{B1}{2a+2b}
\EdgeR{A2}{B2}{b} \LoopS[.5]{A2}{2a+2b} \LoopS[.5]{B2}{4a+4b}
\EdgeR{A1}{A2}{b} \EdgeL{B1}{B2}{b} \EdgeL{A1}{B2}{b}
%
\end{VCPicture}
\end{verbatim}
\normalsize
\end{minipage}% 

\noi
The \verb+\VCPut+ command can also be used to rotate a figure by using
an optional parameter.


\noi 
\hspace*{-\jsIndent}
\begin{minipage}[t]{\ColoText}
        \par\vspace*{0mm}% pour l'alignement sur le haut
        \footnotesize
\verb+\VCPut[+$r$\verb+]{(+$x$,$y$\verb+)}{+ instructions\ldots \verb+}+
\normalsize
\end{minipage}% 
\hspace*{1.2em}% 
\begin{minipage}[t]{\ColoFigu}%
% \setlength{\parindent}{\parindenttemp}%
\par\vspace*{0mm}% pour l'alignement sur le haut
Rotate the result of the instructions with angle~$r$ and center~$(0,0)$
and then translate it by the vector~($x$, $y$).
\end{minipage}%

\medskip
\noi 
\hspace*{-\jsIndent}
\begin{minipage}[c]{\ColFigur}%
\par\vspace*{0mm}% pour l'alignement sur le haut
\begin{center}
%       \ShowFrame 
\SmallPicture\VCDraw{%
\begin{VCPicture}{(-1,-1)(5,4)}
\StateVar[(0,0)]{(0,0)}{O}
\State[A]{(3,0)}{A}
\VCPut[90]{(0,0)}{\StateVar[90^o]{(3,0)}{B}}
\VCPut[90]{(4,0)}{\State[C]{(3,0)}{C}}
\LArcR{A}{B}{r} \EdgeL{B}{C}{t}
\end{VCPicture}
}
\end{center}
\end{minipage}%
\hspace*{1.2em}% 
\begin{minipage}[c]{\ColSource}
\setlength{\parindent}{\parindenttemp}%
\par\vspace*{0mm}% pour l'alignement sur le haut
\footnotesize
\begin{verbatim}
\begin{VCPicture}{(-1,-1)(5,4)}
\StateVar[(0,0)]{(0,0)}{O}
\State[A]{(3,0)}{A}
\VCPut[90]{(0,0)}{\StateVar[90^o]{(3,0)}{B}}
\VCPut[90]{(4,0)}{\State[C]{(3,0)}{C}}
\LArcR{A}{B}{r} \EdgeL{B}{C}{t}
\end{VCPicture}
\end{verbatim}
\normalsize
\end{minipage}% 

The default option is that labels of states (except those of \emph{variable states}) are put in a horizontal
position (\verb+\RigidLabel+), but the command \verb+\SwivelLabel+
makes them rotate with the figure.\index{RigidLabel}\index{SwivelLabel}

\medskip
\noi 
\hspace*{-\jsIndent}
\begin{minipage}[c]{\ColFigur}%
\par\vspace*{0mm}% pour l'alignement sur le haut
\begin{center}
%       \ShowFrame 
\SmallPicture\VCDraw{%
\begin{VCPicture}{(-5,-5)(5,5)}
\SwivelLabel
\State[1]{(4,0)}{A1}\LoopE{A1}{a}
\VCPut[40]{(0,0)}{\State[2]{(4,0)}{A2}\LoopE{A2}{a}\ArcR{A1}{A2}{b}}
\VCPut[80]{(0,0)}{\State[3]{(4,0)}{A3}\LoopE{A3}{a}\ArcR{A2}{A3}{b}}
\VCPut[120]{(0,0)}{\State[4]{(4,0)}{A4}\LoopE{A4}{a}}
\VCPut[160]{(0,0)}{\State[5]{(4,0)}{A5}\LoopE{A5}{a}}
\RigidLabel
\VCPut[200]{(0,0)}{\State[6]{(4,0)}{A6}\LoopE{A6}{a}\ArcR{A5}{A6}{b}}
\VCPut[240]{(0,0)}{\State[7]{(4,0)}{A7}\LoopE{A7}{a}\ArcR{A6}{A7}{b}}
\VCPut[280]{(0,0)}{\State[8]{(4,0)}{A8}}
\VCPut[320]{(0,0)}{\State[9]{(4,0)}{A9}}
\ArcR{A3}{A4}{b} \ArcR{A4}{A5}{b}
\ArcR{A7}{A8}{b} \ArcR{A8}{A9}{b} \ArcR{A9}{A1}{b}
\LoopL{280}{A8}{a} \LoopL{320}{A9}{a}
\end{VCPicture}
}%
\end{center}
\end{minipage}%
\hspace*{1.2em}% 
\begin{minipage}[c]{\ColSource}
\setlength{\parindent}{\parindenttemp}%
\par\vspace*{0mm}% pour l'alignement sur le haut
\footnotesize
\begin{verbatim}
\begin{VCPicture}{(-5,-5)(5,5)}
\SwivelLabel
\State[1]{(4,0)}{A1}\LoopE{A1}{a}
\VCPut[40]{(0,0)}{%
   \State[2]{(4,0)}{A2}\LoopE{A2}{a}\ArcR{A1}{A2}{b}}
\VCPut[80]{(0,0)}{%
   \State[3]{(4,0)}{A3}\LoopE{A3}{a}\ArcR{A2}{A3}{b}}
\VCPut[120]{(0,0)}{\State[4]{(4,0)}{A4}\LoopE{A4}{a}}
\VCPut[160]{(0,0)}{\State[5]{(4,0)}{A5}\LoopE{A5}{a}}
\RigidLabel
\VCPut[200]{(0,0)}{%
   \State[6]{(4,0)}{A6}\LoopE{A6}{a}\ArcR{A5}{A6}{b}}
\VCPut[240]{(0,0)}{%
   \State[7]{(4,0)}{A7}\LoopE{A7}{a}\ArcR{A6}{A7}{b}}
\VCPut[280]{(0,0)}{\State[8]{(4,0)}{A8}}
\VCPut[320]{(0,0)}{\State[9]{(4,0)}{A9}}
\ArcR{A3}{A4}{b} \ArcR{A4}{A5}{b}
\ArcR{A7}{A8}{b} \ArcR{A8}{A9}{b} \ArcR{A9}{A1}{b}
\LoopL{A8}{280}{a} \LoopL{A9}{320}{a}
\end{VCPicture}
\end{verbatim}
\normalsize
\end{minipage}% 

One can notice that \verb+\RigidLabel+ only acts on labels of states.
To draw arcs with ``rigid'' labels, commands have to be written outside
\verb+\VCPut+. We shall see later how to deal with the command \verb+\LoopL+
in order to give any orientation to loops without using \verb+\VCPut+.


%%%%%%%%%%%%%%%%%%%%%%
\subsection{Using separate files}\label{subsec.fil}

In order to use several times the same figure in the same or in 
different papers, or even, as we shall see in the next section, in two 
different contexts such as in a paper and in a slide for a talk,
it is convenient to have the figure written into a file and imported
in the main \LaTeX \ source file.
This can be handled by the following macros.

\noi 
\hspace*{-\jsIndent}
\begin{minipage}[t]{\ColoText}
        \par\vspace*{0mm}% pour l'alignement sur le haut
        \footnotesize
\verb+\VCCall{+\textsl{FileName}\verb+}+
\normalsize
\end{minipage}% 
\hspace*{1.2em}% 
\begin{minipage}[t]{\ColoFigu}%
% \setlength{\parindent}{\parindenttemp}%
\par\vspace*{0mm}% pour l'alignement sur le haut
Calls the file \textsl{FileName} and apply the~\verb+\VCScale+.\index{VCCall}
\end{minipage}%

\noi 
\hspace*{-\jsIndent}
\begin{minipage}[t]{\ColoText}
        \par\vspace*{0mm}% pour l'alignement sur le haut
        \footnotesize
\verb+\SetVCDirectory{+\textsl{Directory}\verb+}+
\normalsize
\end{minipage}% 
\hspace*{1.2em}% 
\begin{minipage}[t]{\ColoFigu}%
% \setlength{\parindent}{\parindenttemp}%
\par\vspace*{0mm}% pour l'alignement sur le haut
Sets to \textsl{Directory} the directory where the file 
\textsl{FileName} will be taken by the macro \verb+\VCCall+ .
Set to the current directory by default.
Usual Unix syntax is used to name directories.
\end{minipage}%

\medskip
\noi 
\hspace*{-\jsIndent}
\begin{minipage}[t]{\ColoText}
        \par\vspace*{0mm}% pour l'alignement sur le haut
        \footnotesize
\verb+\ShowName+ \ee 
\verb+\HideName+ 
\normalsize
\end{minipage}% 
\hspace*{1.2em}% 
\begin{minipage}[t]{\ColoFigu}%
% \setlength{\parindent}{\parindenttemp}%
\par\vspace*{0mm}% pour l'alignement sur le haut
Sets or unsets a flag that controls the printing of  
\textsl{FileName} besides the figure drawn by \verb+\VCCall+.
\index{ShowName}\index{HideName}
\end{minipage}%

\medskip
\noi
For the example below, the following lines have been saved in the 
file \verb+TwoStates.tex+:
\small
\begin{verbatim}
\begin{VCPicture}{(-1,-1)(4,1)}
\State[p]{(0,0)}{A} \State[q]{(3,0)}{B}%
\ArcL{A}{B}{b} \LArcL{B}{A}{a} \LoopE{B}{a}
\end{VCPicture}
\end{verbatim}
\normalsize 

\verb+\VCCall+ applies the same parameters to figures described
in files as \verb+\VCDraw+ does. 

\noi 
\hspace*{-\jsIndent}
\begin{minipage}[c]{\ColFigur}%
\par\vspace*{0mm}% pour l'alignement sur le haut
% \begin{center}
\ee \VCCall{TwoStates}

\medskip

\SmallPicture \ShowName 
\ee \VCCall{TwoStates}
% \end{center}
\end{minipage}%
\hspace*{1.2em}% 
\begin{minipage}[c]{\ColSource}
\setlength{\parindent}{\parindenttemp}%
\par\vspace*{0mm}% pour l'alignement sur le haut
\footnotesize
\begin{verbatim}
\VCCall{TwoStates}

\SmallPicture \ShowName 
\VCCall{TwoStates}
\end{verbatim}
\normalsize
\end{minipage}% 

%%%%%%%%%%%%%%%%%%%%%%
\subsection{Frame and grid}\label{subsec.ffra}

% %%%%%%%%%%%%%%%%%%%%%%
% \paragraph{Frame and grid}
% ~
Two macros that help placing the figure in the page, and the states in 
the figure.  

\noi 
\hspace*{-\jsIndent}
\begin{minipage}[t]{\ColoText}
        \par\vspace*{0mm}% pour l'alignement sur le haut
        \footnotesize
\verb+\ShowFrame+ \ee \verb+\HideFrame+ \ee
\end{minipage}% 
\hspace*{1.2em}% 
\begin{minipage}[t]{\ColoFigu}%
% \setlength{\parindent}{\parindenttemp}%
\par\vspace*{0mm}% pour l'alignement sur le haut
Sets or unsets a flag that controls the drawing of the \emph{bounding box} which is 
defined by the parameter of the \verb+\VCPicture+ environment.
\index{ShowFrame}\index{HideFrame}
\end{minipage}%

\medskip
\noi 
\hspace*{-\jsIndent}
\begin{minipage}[t]{\ColoText}
        \par\vspace*{0mm}% pour l'alignement sur le haut
        \footnotesize
\verb+\ShowGrid+ \ee \verb+\HideGrid+ \ee 
   \end{minipage}% 
\hspace*{1.2em}% 
\begin{minipage}[t]{\ColoFigu}%
% \setlength{\parindent}{\parindenttemp}%
\par\vspace*{0mm}% pour l'alignement sur le haut
Sets or unsets a flag that controls the drawing of a \emph{grid} inside the 
bounding box which is  
defined by the parameter of the \verb+\VCPicture+ environment.
\index{ShowGrid}\index{HideGrid}
\end{minipage}%


\medskip
\noi 
Default settings are \verb+\HideFrame+  and  \verb+\HideGrid+ .

\noi 
\hspace*{-\jsIndent}
\begin{minipage}[t]{\ColFigur}%
\par\vspace*{0mm}% pour l'alignement sur le haut
% \begin{center}
\ee \ShowFrame  \ShowGrid \VCCall{TwoStates}

% \end{center}
\end{minipage}%
\hspace*{1.2em}% 
\begin{minipage}[t]{\ColSource}
\setlength{\parindent}{\parindenttemp}%
\par\vspace*{0mm}% pour l'alignement sur le haut
\medskip
\footnotesize
\begin{verbatim}
\ShowFrame \ShowGrid \VCCall{TwoStates}
\end{verbatim}
\normalsize
\end{minipage}% 


\subsection{The grid scale}
This parameter controls the value of the \PSTricks unit length.
As values that define the size of states and the width of lines are given with fixed units
(as \texttt{cm} or \texttt{pt}), they are not modified by this parameter.
However, the scale of the grid on which they are put, \ie the
distance between each other is modified by grid scale.
This parameter is actually the first (optional) argument of \verb+\VCDraw+
or \verb+\VCCall.+

The following example is the figure of Subsection~\ref{subsec.ex1}
with another grid scale:

\noi 
\hspace*{-\jsIndent}
\begin{minipage}[c]{\ColFigur}%
\par\vspace*{0mm}% pour l'alignement sur le haut
\begin{center}
\VCDraw[.7]{%
\begin{VCPicture}{(0,-2)(6,2)}
% states
\State[i]{(0,0)}{A} \State[q]{(3,0)}{B} \State[t]{(6,0)}{C}
% initial-final
\Initial{A} \Final{C}
% transitions 
\EdgeR{A}{B}{2b} \EdgeR{B}{C}{2b} \LArcL{A}{C}{b}
\LoopN{A}{a} \LoopS{A}{b} \LoopS[.5]{B}{2a+2b} 
\LoopN[.75]{C}{4a} \LoopS[.75]{C}{4b}
%
\end{VCPicture}%
}%
\end{center}
\end{minipage}%
\hspace*{1.2em}% 
\begin{minipage}[c]{\ColSource}
\setlength{\parindent}{\parindenttemp}%
\par\vspace*{0mm}% pour l'alignement sur le haut
\footnotesize
\begin{verbatim}
\VCDraw[.7]{%
\begin{VCPicture}{(0,-2)(6,2)}
% states
\State[i]{(0,0)}{A} \State[q]{(3,0)}{B} \State[t]{(6,0)}{C}
% initial-final
\Initial{A} \Final{C}
% transitions 
\EdgeR{A}{B}{2b} \EdgeR{B}{C}{2b} \LArcL{A}{C}{b}
\LoopN{A}{a} \LoopS{A}{b} \LoopS[.5]{B}{2a+2b} 
\LoopN[.75]{C}{4a} \LoopS[.75]{C}{4b}
\end{VCPicture}
}
\end{verbatim}
\normalsize
\end{minipage}% 

We can call \verb+TwoStates+ with another grid scale.

\noi 
\hspace*{-\jsIndent}
\begin{minipage}[t]{\ColFigur}%
\par\vspace*{0mm}% pour l'alignement sur le haut
% \begin{center}
\ee \ShowFrame  \ShowGrid \VCCall[1.5]{TwoStates}

% \end{center}
\end{minipage}%
\hspace*{1.2em}% 
\begin{minipage}[t]{\ColSource}
\setlength{\parindent}{\parindenttemp}%
\par\vspace*{0mm}% pour l'alignement sur le haut
\medskip
\bigskip
\footnotesize
\begin{verbatim}
\ShowFrame \ShowGrid \VCCall[1.5]{TwoStates}
\end{verbatim}
\normalsize
\end{minipage}% 

%%%%%%%%%%%%%%%%%%%%%%
\section{Parametrized drawing commands}\label{sec.par}

\noi
In the sequel, the letter \texttt{X} stands for either 
\texttt{L} or \texttt{R}.

\subsection{Parametrized loops}
One can draw loops around state with any direction.

\noi 
\hspace*{-\jsIndent}
\begin{minipage}[t]{\ColoText}
        \par\vspace*{0mm}% pour l'alignement sur le haut
\medskip 
        \footnotesize
\verb+\LoopX[+\textsl{Pos}\verb+]{+%
   \textsl{Direction}\verb+}{+%
   \textsl{Name}\verb+}{+%
   \textsl{Label}\verb+}+
   
\medskip 
\verb+\CLoopX[+\textsl{Pos}\verb+]{+%
   \textsl{Direction}\verb+}{+%
   \textsl{Name}\verb+}{+%
   \textsl{Label}\verb+}+
\end{minipage}% 
\hspace*{1.2em}% 
\begin{minipage}[t]{\ColoFigu}%
% \setlength{\parindent}{\parindenttemp}%
\par\vspace*{0mm}% pour l'alignement sur le haut
Draw a loop around the state {\sl Name} in direction {\sl Direction},
which is an angle with respect to east, with label {\sl Label}.

As the label 
is placed outside the loop, \verb+\LoopL+ is reversed
with respect to \verb+\LoopR+.
The opening of the loop is~$60^o$ for the \verb+LoopX+ commands,
$44^o$ for the \verb+CLoopX+ commands.\index{Loop}\index{CLoop}
\end{minipage}%

\medskip 
\noi 
\hspace*{-\jsIndent}
\begin{minipage}[c]{\ColFigur}%
\par\vspace*{0mm}% pour l'alignement sur le haut
\begin{center}
\SmallPicture\VCDraw{%
\begin{VCPicture}{(0,-2)(6,2)}
% states
\State[i]{(0,0)}{A} \State[q]{(3,0)}{B} \State[t]{(6,0)}{C}
% initial-final
\Initial{A} \Final{C}
% transitions 
\EdgeR{A}{B}{2b} \EdgeR{B}{C}{2b} \LArcL{A}{C}{b}
\LoopR{120}{A}{a} \LoopL{-120}{A}{b} \LoopS[.5]{B}{2a+2b} 
\LoopL[.5]{60}{C}{4a} \LoopR[.5]{-60}{C}{4b}
%
\end{VCPicture}%
}
\end{center}
\end{minipage}%
\hspace*{1.2em}% 
\begin{minipage}[c]{\ColSource}
\setlength{\parindent}{\parindenttemp}%
\par\vspace*{0mm}% pour l'alignement sur le haut
\footnotesize
\begin{verbatim}
\begin{VCPicture}{(0,-2)(6,2)}
% states
\State[i]{(0,0)}{A} \State[q]{(3,0)}{B} \State[t]{(6,0)}{C}
% initial-final
\Initial{A} \Final{C}
% transitions 
\EdgeR{A}{B}{2b} \EdgeR{B}{C}{2b} \LArcL{A}{C}{b}
\LoopR{120}{A}{a} \LoopL{-120}{A}{b} \LoopS[.5]{B}{2a+2b} 
\LoopL[.5]{60}{C}{4a} \LoopR[.5]{-60}{C}{4b}
%
\end{VCPicture}%
\end{verbatim}
\normalsize
\end{minipage}% 

\subsection{Parametrized arcs}

\noi 
\hspace*{-\jsIndent}
\begin{minipage}[c]{\ColoText+9em}
        \par\vspace*{0mm}% pour l'alignement sur le haut
% \medskip 
        \footnotesize
\verb+\VArcX[+\textsl{Pos}\verb+]{arcangle=+%
   \textsl{x}\verb+,ncurv=+%
   \textsl{v}\verb+}{+%
   \textsl{Start}\verb+}{+%
   \textsl{End}\verb+}{+%
   \textsl{Label}\verb+}+
\end{minipage}% 
  \begin{minipage}[c]{\ColoFigu-9em}%
% \setlength{\parindent}{\parindenttemp}%
\par\vspace*{0mm}% pour l'alignement sur le haut
% \medskip
% \bigskip
Draw an arc from \textsl{Start} to \textsl{End} with label 
\textsl{Label}, with a starting angle \texttt{x} and an eccentricity 
\texttt{v}. 
\index{VArc}
\end{minipage}%

\noi
The starting angle is defined by the parameter \verb+arcangle=+$x$
and is measured, going to the left if 
\texttt{X=L}, to the right if \texttt{X=R}, from the line that goes 
from \textsl{Start} to \textsl{End}, called the \emph{base line}.
% with an angle defined respectively by \texttt{arcangle}.
The parameter \verb+ncurv+ is the `eccentricity' of the arc and some 
how measures the distance between the center of the arc and the base 
line as one can observe on the figure below.
\index{eccentricity (of curve)}

The first (non optional) parameter of \verb+\VArcX+ contains 
PSTrick's parameters, written with PSTrick's syntax: they are 
separated by a comma and there must be no space inside the braces 
that contains them.  
All these parameters are optional, but this first parameter of 
\verb+\VArcX+ itself is not an optional argument, which 
means that this command must always have
four arguments (and possibly an optional one): 

\begin{minipage}[c]{6cm}
\verb+\VArcL{arcangle=15}{A}{B}{x}+\\
\verb+\VArcL{ncurve=1}{A}{B}{x}+\\
\verb+\VArcL{}{A}{B}{x}+
\end{minipage}
\ are syntactically correct commands.

\noi 
Default value for \verb+arcangle+
is the same as for \verb+\LArc+, whereas the default \verb+ncurv+
value is~$1$. 

\medskip
\noi 
\hspace*{-\jsIndent}
\begin{minipage}[c]{\ColFigur}%
\par\vspace*{0mm}% pour l'alignement sur le haut
\begin{center}
\scalebox{\VCScale}{%
\begin{VCPicture}{(-3,-2)(3,6)}
%
\LargeState
\State[A]{(-2,0)}{A}
\State[B]{(2,0)}{B}
%transitions
\VArcR[.5]{arcangle=60,ncurv=.2}{A}{B}{0.2}
\VArcL[.5]{arcangle=60,ncurv=.8}{A}{B}{0.8}
\VArcL[.5]{arcangle=60,ncurv=2}{A}{B}{2}
\VArcR[.5]{arcangle=60,ncurv=6}{A}{B}{6}
\VArcL{arcangle=-60}{A}{B}{60}
\VArcR{arcangle=-90}{A}{B}{90}
\VArcR[.2]{arcangle=-120}{A}{B}{120}
\end{VCPicture}%
}%
\end{center}
\end{minipage}%
\hspace*{1.2em}% 
\begin{minipage}[c]{\ColSource}
\setlength{\parindent}{\parindenttemp}%
\par\vspace*{0mm}% pour l'alignement sur le haut
\footnotesize
\begin{verbatim}
\begin{VCPicture}{(-3,-2)(3,6)}
% states
\LargeState
\State[A]{(-2,0)}{A}
\State[B]{(2,0)}{B}
%transitions
\VArcR[.5]{arcangle=45,ncurv=.2}{A}{B}{0.2}
\VArcL[.5]{arcangle=45,ncurv=.8}{A}{B}{0.8}
\VArcL[.5]{arcangle=45,ncurv=2}{A}{B}{2}
\VArcR[.5]{arcangle=45,ncurv=6}{A}{B}{6}
\VArcL{arcangle=-60}{A}{B}{60}
\VArcR{arcangle=-90}{A}{B}{90}
\VArcR[.2]{arcangle=-120}{A}{B}{120}
%
\end{VCPicture}
\end{verbatim}
\normalsize
\end{minipage}% 


\subsection{General curve}

\noi 
\hspace*{-\jsIndent}
\begin{minipage}[c]{\ColoText+12em}
        \par\vspace*{0mm}% pour l'alignement sur le haut
% \medskip 
        \footnotesize
\verb+\VCurveX[+\textsl{Pos}\verb+]{angleA=+%
   \textsl{x}\verb+,angleB=+%
   \textsl{y}\verb+,ncurv=+%
   \textsl{v}\verb+}{+%
   \textsl{Start}\verb+}{+%
   \textsl{End}\verb+}{+%
   \textsl{Label}\verb+}+
   
\end{minipage}% 
\hspace*{1.2em}% 
\begin{minipage}[c]{\ColoFigu-12em}%
% \setlength{\parindent}{\parindenttemp}%
\par\vspace*{0mm}% pour l'alignement sur le haut
% \medskip
% \bigskip
Draw a curve from \textsl{Start} to \textsl{End} with label 
\textsl{Label}, starting with an angle \textsl{x}, arriving with an 
angle \textsl{y} and with an eccentricity \textsl{v}.
\end{minipage}%
\index{VCurve}
\index{angleA}
\index{angleB}

\medskip
\noi
A (B\'ezier) curve, computed by \PSTricks with the three parameters:
the \emph{starting angle} fixed by \verb+angleA=+\textsl{x}, the 
ending angle fixed by \verb+angleB=+\textsl{y}, and the eccentricity 
\verb+ncurv=+\textsl{v}.
The two angles are measured, as trigonometric angles are, from
the East direction and counterclockwise. 
The syntax of these parameters is \PSTricks's and they are optional, 
as in \verb+\VArcX+.
Default values for \verb+angleA+
and \verb+angleB+ are respectively $0$ and $180$; \verb+ncurv+ is fixed
by default to $1$.

\medskip
\noi 
\hspace*{-\jsIndent}
\begin{minipage}[c]{\ColFigur}%
\par\vspace*{0mm}% pour l'alignement sur le haut
\begin{center}
\scalebox{\VCScale}{%
\begin{VCPicture}{(-4,-4)(4,4)}
%
\State[A]{(-3,0)}{A}
\State[B]{(0,0)}{B}
\State[C]{(3,0)}{C}
%transitions
\VCurveL{angleA=90,angleB=135}{A}{B}{1}
\VCurveR[.5]{angleA=90,angleB=135,ncurv=.5}{A}{B}{0.5}
\VCurveL{angleA=90,angleB=135,ncurv=2}{A}{B}{2}
\VCurveR[.6]{angleA=-80,angleB=175}{A}{B}{b}
\VCurveR[.6]{angleA=-80,angleB=175}{A}{C}{c}
\VCurveL{angleB=20,angleB=160}{B}{C}{d}
\end{VCPicture}%
}%
\end{center}
\end{minipage}%
\hspace*{1.2em}% 
\begin{minipage}[c]{\ColSource}
\setlength{\parindent}{\parindenttemp}%
\par\vspace*{0mm}% pour l'alignement sur le haut
\footnotesize
\begin{verbatim}
\begin{VCPicture}{(-4,-4)(4,4)}
%states
\State[A]{(-3,0)}{A}
\State[B]{(0,0)}{B}
\State[C]{(3,0)}{C}
%transitions
\VCurveL{angleA=90,angleB=135}{A}{B}{1}
\VCurveR[.5]{angleA=90,angleB=135,ncurv=.5}{A}{B}{0.5}
\VCurveL{angleA=90,angleB=135,ncurv=2}{A}{B}{2}
\VCurveR[.6]{angleA=-80,angleB=175}{A}{B}{b}
\VCurveR[.6]{angleA=-80,angleB=175}{A}{C}{c}
\VCurveL{angleB=20,angleB=160}{B}{C}{d}
\end{VCPicture}
\end{verbatim}
\normalsize
\end{minipage}% 


\subsection{Transitions labelling}

\paragraph{Initial and final arrows with a label} When one deals with
tranducers, or automata with multiplicity, one may need to put
some labels on initial or final arrows.

\medskip 
\noi 
\hspace*{-\jsIndent}
\begin{minipage}[t]{\ColoText}
        \par\vspace*{0mm}% pour l'alignement sur le haut
        \footnotesize
\verb+\InitialL[+\textsl{Pos}\verb+]{+%
   \textsl{Dir}\verb+}{+\textsl{Name}\verb+}{+\textsl{Label}\verb+}+
\\ \verb+\InitialR[+\textsl{Pos}\verb+]{+%
   \textsl{Dir}\verb+}{+\textsl{Name}\verb+}{+\textsl{Label}\verb+}+
\normalsize
\end{minipage}% 
\hspace*{1.2em}% 
\begin{minipage}[t]{\ColoFigu}%
% \setlength{\parindent}{\parindenttemp}%
\par\vspace*{0mm}% pour l'alignement sur le haut
Makes the state \textsl{Name} initial and writes \textsl{Label}
on the left or the right size of the incoming arrow.\index{Initial}\index{Final}
\end{minipage}%

\medskip 
\noi 
\hspace*{-\jsIndent}
\begin{minipage}[t]{\ColoText}
        \par\vspace*{0mm}% pour l'alignement sur le haut
        \footnotesize
\verb+\FinalL[+\textsl{Pos}\verb+]{+%
   \textsl{Dir}\verb+}{+\textsl{Name}\verb+}{+\textsl{Label}\verb+}+
\\ \verb+\FinalR[+\textsl{Pos}\verb+]{+%
   \textsl{Dir}\verb+}{+\textsl{Name}\verb+}{+\textsl{Label}\verb+}+
\normalsize
\end{minipage}% 
\hspace*{1.2em}% 
\begin{minipage}[t]{\ColoFigu}%
% \setlength{\parindent}{\parindenttemp}%
\par\vspace*{0mm}% pour l'alignement sur le haut
Makes the state \textsl{Name} final and writes \textsl{Label}
on the left or the right size of the outgoinging arrow.
\end{minipage}%


\medskip 
\noi 
Here, the parameter \textsl{Dir} is not optional. Whereas, there is
an optional parameter \textsl{Pos} that defines the position
of the label on the arrow. Its default values is fixed to $0.1$ ({\it resp.} $0.9$)
for initial ({\it resp.} final) arrows. The label is thus written far enough of the state.


\paragraph{Independant labelling of transitions}
\medskip
There exists commands to label transitions that
have not been labeled yet or to put a second label on them.
These commands must directly follow the definition of
the transition that will be labeled.

\noi 
\hspace*{-\jsIndent}
\begin{minipage}[t]{\ColoText}
        \par\vspace*{0mm}% pour l'alignement sur le haut
        \footnotesize
\verb+\LabelL[+%
   \textsl{Pos}\verb+]{+%
   \textsl{Label}\verb+}+
   
\medskip 
\verb+\LabelR[+%
   \textsl{Pos}\verb+]{+%
   \textsl{Label}\verb+}+
\end{minipage}% 
\hspace*{1.2em}% 
\begin{minipage}[t]{\ColoFigu}%
% \setlength{\parindent}{\parindenttemp}%
\par\vspace*{0mm}% pour l'alignement sur le haut
Puts a label on the previous transition. The default position of the label
is the same as for edges.\index{Label}
\end{minipage}%


\medskip
\noi 
\hspace*{-\jsIndent}
\begin{minipage}[c]{\ColFigur}%
\par\vspace*{0mm}% pour l'alignement sur le haut
\begin{center}
%       \ShowFrame 
%\SmallPicture
\VCDraw{%
\begin{VCPicture}{(-3,-2)(3,2)}
\State[A]{(-2,1)}{A} \State[B]{(2,1)}{B}
\State[C]{(-2,-1)}{C} \State[D]{(2,-1)}{D}
\EdgeL{A}{B}{u} \LabelR{\varphi(u)=3}
\LArcR{D}{C}{} \LabelL{v}
\VCurveL[.3]{angleA=45,angleB=-45}{B}{D}{a}
\LabelL[.5]{b} \LabelL[.7]{c}
%
\end{VCPicture}
}%
\end{center}
\end{minipage}%
\hspace*{1.2em}% 
\begin{minipage}[c]{\ColSource}
\setlength{\parindent}{\parindenttemp}%
\par\vspace*{0mm}% pour l'alignement sur le haut
\footnotesize
\begin{verbatim}
\begin{VCPicture}{(-3,-2)(3,2)}
\State[A]{(-2,1)}{A} \State[B]{(2,1)}{B}
\State[C]{(-2,-1)}{C} \State[D]{(2,-1)}{D}
%
\EdgeL{A}{B}{u} \LabelR{\varphi(u)=3}
\LArcR{D}{C}{} \LabelL{v}
\VCurveL[.3]{angleA=45,angleB=-45}{B}{D}{a}
\LabelL[.5]{b} \LabelL[.7]{c}
\end{VCPicture}
\end{verbatim}
\normalsize
\end{minipage}% 

\subsection{Miscellaneous}


\paragraph{Loops on variable states}
Instead of using \verb+\VarLoopN+ or \verb+\VarLoopS+, one can modify
parameters of loops to draw, for instance, many loops on variable states.
As usual, these loops may be drawn on the south or north side of
the states only.

\noi 
\hspace*{-\jsIndent}
\begin{minipage}[t]{\ColoText}
        \par\vspace*{0mm}% pour l'alignement sur le haut
        \footnotesize
\verb+\VarLoopOn+
   
\medskip 
\verb+\VarLoopOff+

\medskip 
\end{minipage}% 
\hspace*{1.2em}% 
\begin{minipage}[t]{\ColoFigu}%
% \setlength{\parindent}{\parindenttemp}%
\par\vspace*{0mm}% pour l'alignement sur le haut
Modifies parameters in order to draw loops around variable states.
\index{VarLoopOn}\index{VarLoopOff}

\smallskip 

Restore the normal style of loops.
\end{minipage}%

\noi
Command \verb+\LoopVarN+ is equivalent to the sequence
\verb+\LoopVarOn \LoopN \LoopVarOff+.

\paragraph{Saving memory}

In default mode, the call to a \verb+\State+ command 
defines eight additional points, that allow then to draw initial and 
final arrows in the prescribed compass direction.
In case of large figures, with many states, this ought to lead
to an overflow of \TeX\   memory. This is the reason why one can define
states without any additional point.

\noi 
\hspace*{-\jsIndent}
\begin{minipage}[t]{\ColoText}
        \par\vspace*{0mm}% pour l'alignement sur le haut
        \footnotesize
\verb+\PlainState+
   
\medskip 
\verb+\FullState+ 
   
\medskip 
\end{minipage}% 
\hspace*{1.2em}% 
\begin{minipage}[t]{\ColoFigu}%
% \setlength{\parindent}{\parindenttemp}%
\par\vspace*{0mm}% pour l'alignement sur le haut
Draw states with no additional points.\index{PlainState}

\smallskip 

Draw states with eight points to draw initial or final arrows.\index{FullState}
\end{minipage}%

Of course, when in \verb+\PlainState+ mode, one can define states
with additional points.

\noi 
\hspace*{-\jsIndent}
\begin{minipage}[t]{\ColoText}
        \par\vspace*{0mm}% pour l'alignement sur le haut
        \footnotesize
\verb+\IFState[+\textsl{Label}\verb+]{+%
   $(x,y)$\verb+}{+%
   \textsl{Name}\verb+}+
   
\medskip 
\verb+\IFXState[+\textsl{Label}\verb+]{+%
   $(x,y)$\verb+}{+%
   \textsl{Name}\verb+}+
   
\medskip 
\end{minipage}% 
\hspace*{1.2em}% 
\begin{minipage}[t]{\ColoFigu}%
% \setlength{\parindent}{\parindenttemp}%
\par\vspace*{0mm}% pour l'alignement sur le haut
Draws a state with four points ($n$, $s$, $e$, $w$).\index{IFState}

\smallskip 

Draws a state with eight points ($n$, $s$, $e$, $w$, $ne$, $se$, $nw$, $sw$).
\index{IFXState}
\end{minipage}%

In \verb+\PlainState+ mode, one should not apply an \verb+\Initial+
or a \verb+\Final+ command to a state defined by \verb+\State+.
Likewise, these commands can be only used for \verb+\IFState+
with directions $n$, $s$, $e$ and $w$.

Note that even when in \verb+\FullState+ mode, 
variable states defined by \verb+\StateVar+ have
 the same four additional points as \verb+\IFState+.


%%%%%%%%%%%%%%%%%%%%%%
\section{Modifying and setting the parameters}\label{sec.mod}

The reason why the commands in \VCSG are simple and concise is that 
all the parameters that control the geometry (size, angles, 
\ldots) and the drawing (line width, line style, line color, \ldots) 
of the states and transitions in an automaton are preset by \VCSG and 
can be kept implicit.
But there are two instances where a user may want to modify these 
preset values.

The first case is when one needs to tune a parameter in order to make 
the figure more readable, or nicer. 
Such a case has already been considered in 
sections~\ref{subsec.edg}, \ref{subsec.arc} and~\ref{subsec.loo}, 
where the parameter \textsl{Pos}   
that controls the position of the label of a transition along the 
edge has been modified with an optional argument. 
The parameters that will be considered in this section are not 
optional parameters of already defined commands but parameters that 
are set with their own commands.

The other case where a user will change the implicit parameters of 
a figure or, more likely, of a set of figures, is when he (or she) 
will be in the position to change completely the ``style'' of the 
drawing.
For instance, a same set of figures may be used in a paper \emph{and} in a 
series of slides for the presentation of the paper.
Of course, the figures have to be set at a larger scale, probably 
in colour, and possibly with different relative sizes of the elements.

We begin with the description of the options for the latter case, 
which is by far simpler. We then explain the syntax convention taken 
to coin the names of the many commands that control the many 
parameters.
Then comes the list of these parameters, for scaling first, then for 
states, for transitions, and finally for the fine tuning of 
some details of the drawing.

%%%%%%%%%%%%%%%%%%%%%%
\subsection{Style files}\label{sec.sty-fil}

The package \VCSG is loaded with default values that are all set up 
in the file \verb+VCPref-default.tex+~. 
\index{VCPref-default}%
\index{VCPref-slides}%
\index{VCPref-beamer}%
\index{VCPref-mystyle}%
The package comes with two other styles, in particular one for 
drawing slides. 
The normal way of using the different style files is to call the 
wrapper \texttt{vaucanson-g} with an option:

\noi 
\hspace*{-\jsIndent}
\begin{minipage}[c]{\ColoText}
        \par\vspace*{0mm}% pour l'alignement sur le haut
        \footnotesize
\verb+\usepackage[+\textsl{option}\verb+]{vaucanson-g}+
\end{minipage}% 
\hspace*{1.2em}% 
\begin{minipage}[c]{\ColoFigu}%
% \setlength{\parindent}{\parindenttemp}%
\par\vspace*{0mm}% pour l'alignement sur le haut
The default option is \texttt{default}; the other possible options are
\texttt{slides}, \texttt{beamer} and \texttt{mystyle}.
%\footnotemark 
\end{minipage}%
% \footnotetext{There exists also, for downward compatibility, a 
% \texttt{beamer} option which calls indeed the same file as 
% \texttt{slides}.}%
% \stepcounter{footnote}%
% \medskip
% \smallskip 

Although it is not the normal usage,
it is also possible to change style in the course of the text by 
loading a style file with an \verb+\input+ command. In this
case, it is recommended to reset all parameters by calling
\verb+VCPref-default.tex+.

\noi 
\hspace*{-\jsIndent}
\begin{minipage}[c]{\ColFigur+6em}%
\par\vspace*{0mm}% pour l'alignement sur le haut
\hspace*{1cm}
\begin{center}
\VCCall{TwoStates}%
\input{VCPref-slides}%
\hspace*{-1cm}%
\VCCall{TwoStates}\\
\vspace*{.5cm}%
\input{VCPref-default}%
\input{VCPref-beamer}%
% \hspace*{1cm}%
\VCCall{TwoStates}\\
\vspace*{.5cm}%
\input{VCPref-default}
\input{VCPref-mystyle}
\VCCall{TwoStates}
\end{center}
\end{minipage}%
\hspace*{1.2em}%
%%%%
\begin{minipage}[c]{\ColSource-6em}
\setlength{\parindent}{\parindenttemp}%
\par\vspace*{0mm}% pour l'alignement sur le haut
% \vspace*{-1cm}
\footnotesize
\begin{verbatim}
\usepackage{vaucanson-g}
...
\VCCall{TwoStates}
...


\usepackage[slides]{vaucanson-g}
...
\VCCall{TwoStates}
...

\usepackage[beamer]{vaucanson-g}
...
\VCCall{TwoStates}
...

...
\input{VCPref-default} % gets back to default values
\input{VCPref-mystyle}
\VCCall{TwoStates}
\end{verbatim}
\normalsize
\end{minipage}% 

\medskip 

\noi
By editing the files \texttt{VCPref-mystyle}, \texttt{VCPref-beamer} 
or \texttt{VCPref-slides}, and giving the  
parameters the values he (or she) thinks more appropriate, a user may 
easily build his (or her) own style for drawing automata. 
Modifying directly \texttt{VCPref-default}, although possible, is not 
necessarily a good idea.
By cut, editing and paste, it is not difficult create many style 
files, and even, with an easy modification of the wrapper 
itself to make them possible options.
The content of these style files are described in the following 
subsections. \texttt{VCPref-slides}, \texttt{VCPref-beamer}
and \texttt{VCPref-mystyle} are 
given at Appendix~B.

%%%%%%%%%%%%%%%%%%%%%%
\subsection{Syntax of parameter settings}
\label{sec.syn}

The set of values of all parameters defines a \emph{style} of figures.
Most of parameters, \eg \verb+\StateLineColor+ which defines the 
color of the line  
that delimits a state, may be given a \emph{temporary} value, within a 
figure, without changing the style.
These parameters are dealt with by means of 3 different commands:

\noi 
% \hspace*{-\jsIndent}
\begin{minipage}[t]{\ColoText}
        \par\vspace*{0mm}% pour l'alignement sur le haut
        \footnotesize
\verb+\Chg+\textsl{Param}\verb+{+ \textsl{Val} \verb+}+
   \end{minipage}% 
\hspace*{1.2em}% 
\begin{minipage}[t]{\ColoFigu}%
% \setlength{\parindent}{\parindenttemp}%
\par\vspace*{0mm}% pour l'alignement sur le haut
Gives the new value \textsl{Val} to \verb+\+\emph{Param}.
\end{minipage}%

% \smallskip
\noi 
% \hspace*{-\jsIndent}
\begin{minipage}[t]{\ColoText}
        \par\vspace*{0mm}% pour l'alignement sur le haut
        \footnotesize
\verb+\Rst+\textsl{Param} 
   \end{minipage}% 
\hspace*{1.2em}% 
\begin{minipage}[t]{\ColoFigu}%
% \setlength{\parindent}{\parindenttemp}%
\par\vspace*{0mm}% pour l'alignement sur le haut
Restores the style value in \verb+\+\emph{Param}.
\end{minipage}%

% \smallskip
\noi 
% \hspace*{-\jsIndent}
\begin{minipage}[t]{\ColoText}
        \par\vspace*{0mm}% pour l'alignement sur le haut
        \footnotesize
\verb+\Set+\textsl{Param}\verb+{+ \textsl{Val} \verb+}+
   \end{minipage}% 
\hspace*{1.2em}% 
\begin{minipage}[t]{\ColoFigu}%
% \setlength{\parindent}{\parindenttemp}%
\par\vspace*{0mm}% pour l'alignement sur le haut
Sets the style value for \verb+\+\emph{Param} to \textsl{Val} .
\end{minipage}%

\medskip
\noi 
% In order to be effective, the command \verb+\Set+\textsl{Param} has to be 
% called \emph{outside} a picture.
The command \verb+\Set+\textsl{Param} can be 
called \emph{outside} a \verb+\VCPicture+ environment, in which case 
it will be effective in all subsequent figures, until a new setting 
of the same parameter.
Its natural place, if it is considered as a modification of the style,
is in the preamble, after the call to the package \VCSG.
If the command \verb+\Set+\textsl{Param} is
called \emph{inside} a \verb+\VCPicture+ environment, its effect will 
be restricted to that figure.

Such parameters will be called \emph{adjustable parameters} in the 
sequel.

Note that in the case where \verb+\+\textsl{Param} is a value
(a \emph{length} or a \emph{scale}), the 
value of \textsl{Val} in \verb+\Chg+\textsl{Param}\verb+{+\textsl{Val}\verb+}+ is a 
\emph{coefficient} which is applied to the style value of \verb+\+\textsl{Param}
defined with \verb+\Set+\textsl{Param}\verb+{+\textsl{Val}\verb+}+.
Therefore, in this case, \verb+\Rst+\textsl{Param} is equivalent to 
\verb+\Chg+\textsl{Param}\verb+{1}+. 

The other parameters, as those that define the ``dimmed'' style,
are called \emph{preset parameters}.
%Belong to that class those parameters that can be ``adjusted'' by means 
%of an optionnal argument, \eg \verb+\EdgeLabelPosit+ which gives 
%the implicit position of the label on an edge.

There is finally a class of parameters that fall in between the 
adjustable and the preset ones, those for which several values are 
``preset'' but that can be given other values. 
%For instance, the scale of the 
%picture  is fixed by the parameter \verb+\VCScale+: it can given 
%the preset values \verb+\NormalScale+, \verb+\SmallScale+ or 
%\verb+\LargeScale+ 
%by means of the command \verb+\NormalPicture+, \verb+\SmallPicture+ 
%or \verb+\LargePicture+ respectively, or given any chosen value 
%\textsl{Val} by the command \verb+\FixVScale{+\textsl{Val}\verb+}+ .
%
These parameters will be called \emph{multivalued preset parameters}.

%%%%%%%%%%%%%%%%%%%%%%
\subsection{Scale parameter {\rm(Multivalued preset parameter)}}\label{sec.sca-par}

\noi 
\hspace*{-\jsIndent}
\begin{minipage}[t]{\ColoText}
        \par\vspace*{0mm}% pour l'alignement sur le haut
        \footnotesize
\verb+\FixVCScale{+\textsl{Scale}\verb+}+
   \end{minipage}% 
\hspace*{1.2em}% 
\begin{minipage}[t]{\ColoFigu}%
% \setlength{\parindent}{\parindenttemp}%
\par\vspace*{0mm}% pour l'alignement sur le haut
Sets \verb+\VCScale+ to \textsl{Scale}.\index{FixVCScale}
\end{minipage}%

\medskip 
\noi 
The commands 
\verb+\MediumPicture+, \verb+\SmallPicture+, \verb+\TinyPicture+ \ 
and \verb+\LargePicture+ set  \verb+\VCScale+ to the values
\verb+\MediumScale+, \verb+\SmallScale+, \verb+\TinyScale+
and \verb+\LargeScale+ respectively.
Their values are
\begin{center}
\begin{tabular}{|c|c|c|c|}
\verb+\TinyScale+ & \verb+\SmallScale+ & \verb+\MediumScale+ & 
\verb+\LargeScale+\\ 
\hline
$0.42$ & $0.5$ &  $0.6$ & $0.85$ \\
\hline
\end{tabular}
\end{center}
%\texttt{0.85} respectively and can be modified with a \verb+\renewcommand+.

These preset values can be modified by a \verb+\renewcommand+.



%%%%%%%%%%%%%%%%%%%%%%
\subsection{State parameters}\label{sec.sta-par}

%%%%%%%%%%%%%%%%%%%%%%
\paragraph{State diameter {\rm(Multivalued preset parameter)}}
~
This parameter controls, \via the commands 
\verb+\MediumState+, \verb+\SmallState+ 
or \verb+\LargeState+ the value of a number of other preset parameters.

\noi 
\hspace*{-\jsIndent}
\begin{minipage}[t]{\ColoText}
        \par\vspace*{0mm}% pour l'alignement sur le haut
        \footnotesize
\verb+\FixStateDiameter{+\textsl{Diam}\verb+}+
   \end{minipage}% 
\hspace*{1.2em}% 
\begin{minipage}[t]{\ColoFigu}%
% \setlength{\parindent}{\parindenttemp}%
\par\vspace*{0mm}% pour l'alignement sur le haut
Sets the state diameter to \textsl{Diam} (\textsl{Diam} is 
a length) until the next call to a \verb+\ZState+ command, or to the 
next figure if called inside a \verb+\VCPicture+.\index{FixStateDiameter}
\end{minipage}%

\smallskip
%\noi 
%\hspace*{-\jsIndent}
%\begin{minipage}[t]{\ColoText}
%       \par\vspace*{0mm}% pour l'alignement sur le haut
%       \footnotesize
%\verb+\setlength{\ZStateDiameter}{+\textsl{ZDiam}\verb+}+
%   \end{minipage}% 
%\hspace*{1.2em}% 
%\begin{minipage}[t]{\ColoFigu}%
%% \setlength{\parindent}{\parindenttemp}%
%\par\vspace*{0mm}% pour l'alignement sur le haut
Preset diameter sizes are:
\begin{center}
\begin{tabular}{|c|c|c|c|}
\verb+\VerySmallState+ & \verb+\SmallState+ & \verb+\MediumState+ & 
\verb+\LargeState+\\ 
\hline
$0.3$cm & $0.6$cm & $0.9$cm &  $1.2$cm\\
\hline
\end{tabular}
\end{center}

These values can be modified by a \verb+\renewcommand+. However, as
commands that call these parameters deal also with other parameters
as the eccentricity of loops or the size of initial and final arrows,
changing these values without modifying the other ones can lead to
an ungainly result.

The shape of a normalized automaton can be drawn in the following way:

\noi 
\hspace*{-\jsIndent}
\begin{minipage}[c]{\ColFigur}%
\par\vspace*{0mm}% pour l'alignement sur le haut
\begin{center}
%\SmallPicture
\VCDraw{%
\begin{VCPicture}{(-5,-3)(5,3)}
\FixStateDiameter{5cm} \ChgStateLabelScale{3}
\State[{\cal A}]{(0,0)}{A}
\MediumState \RstStateLabelScale
\State[i]{(-4,0)}{I} \State[t]{(4,0)}{T}
\VSState{(-1,-1)}{P1} \VSState{(-1,1)}{P2} \VSState{(-1.4,0)}{P3} 
\VSState{(1,-1)}{Q1} \VSState{(1,1)}{Q2} \VSState{(1.4,0)}{Q3}
\Initial{I} \Final{T}
%
\EdgeL{I}{P1}{} \EdgeL{I}{P2}{} \EdgeL{I}{P3}{}
\EdgeL{Q1}{T}{} \EdgeL{Q2}{T}{} \EdgeL{Q3}{T}{}
\end{VCPicture}
}%
\end{center}
\end{minipage}%
\hspace*{1.2em}% 
\begin{minipage}[c]{\ColSource}
\setlength{\parindent}{\parindenttemp}%
\par\vspace*{0mm}% pour l'alignement sur le haut
\footnotesize
\begin{verbatim}
\begin{VCPicture}{(-5,-3)(5,3)}
\FixStateDiameter{5cm} \ChgStateLabelScale{3}
\State[{\cal A}]{(0,0)}{A}
\MediumState \RstStateLabelScale
\State[i]{(-4,0)}{I} \State[t]{(4,0)}{T}
\VSState{(-1,-1)}{P1} \VSState{(1,-1)}{Q1}
\VSState{(-1,1)}{P2} \VSState{(1,1)}{Q2 ]
\VSState{(-1.4,0)}{P3} \VSState{(1.4,0)}{Q3}
\Initial{I} \Final{T}
%
\EdgeL{I}{P1}{} \EdgeL{I}{P2}{} \EdgeL{I}{P3}{}
\EdgeL{Q1}{T}{} \EdgeL{Q2}{T}{} \EdgeL{Q3}{T}{}
\end{VCPicture}
\end{verbatim}
\normalsize
\end{minipage}% 



%%%%%%%%%%%%%%%%%%%%%%
\paragraph{State adjustable parameters}
~ Each following parameter can be modified by the command
\verb+\Chg+\textit{Parameter}. The style value can be restored by 
\verb+\Rst+\textit{Parameter} 
and one can set the style value with \verb+\Set+\textit{Parameter}.

\begin{center}
\begin{tabular}{c|c|c|}
Parameter & Default value & Possible values\\
\hline
\textit{StateLineStyle} & \texttt{solid} & \texttt{dashed}, \texttt{dotted}, \texttt{none}\\
\hline
\textit{StateLineWidth} & \texttt{1.8pt} &\\
\hline
\textit{StateLineColor} & \texttt{black} & \texttt{red}, \texttt{lightgray}, \ldots\\
\hline
\textit{StateLabelColor} & \texttt{black} & \texttt{blue}, \texttt{Salmon}, \ldots\\
\hline
\textit{StateLabelScale} & \texttt{1.7} &\\
\hline
\textit{StateFillStatus} & \texttt{solid} & \texttt{vlines}, \texttt{hlines}, \texttt{crosshatch}, \texttt{none}\\
\hline
\textit{StateFillColor} & \texttt{white} & \texttt{green}, \texttt{Purple}, \ldots\\
\hline
\end{tabular}
\end{center}
\index{StateLineStyle}\index{StateLineWidth}\index{StateLineColor}
\index{StateLabelColor}\index{StateLabelScale}\index{StateFillStatus}
\index{StateFillColor}

As it has been previously noticed, the argument of \verb+\ChgStateLineWidth+
is not a \texttt{length} but a coefficient applied to the style width.
For instance, with the default value of \textit{StateLineWidth}, 
after \verb+\ChgStateLineWidth{2}+
edges will be drawn with a \texttt{3.6pt} width.

When pictures are drawn to scale \verb+\VCScale+ that is set to \texttt{0.6},
printing state labels with scale \texttt{1.7} makes them to be as large
as characters of the text. As for line width, \verb+\ChgStateLabelScale{.5}+
sets the label scale to \texttt{0.85} that make then twice smaller
than characters of the text (because of \verb+\VCScale+).

\clearpage 
%%%%%%%%%%%%%%%%%%%%%%
\paragraph{State preset parameters}
~
They deal with the definition of ``dimmed'' states.
%To be modified with a \verb+\renewcommand+ .

\begin{center}
\begin{tabular}{c|c|c|}
Parameter & Default value\\
\hline
\textit{DimStateLineStyle} & \texttt{solid} \\
\hline
\textit{DimStateLineColor} & \texttt{gray} \\
\hline
\textit{DimStateLineCoef} & \texttt{1} \\
\hline
\textit{DimStateLabelColor} & \texttt{gray}\\
\hline
\textit{DimStateFillColor} & \texttt{white} \\
\hline
\end{tabular}
\end{center}
\index{DimStateLineStyle}\index{DimStateLineColor}\index{DimStateLineCoef}
\index{DimStateLabelColor}\index{DimStateFillColor}

\textit{DimStateLineCoef} is a coefficient that is applied to
\textit{StateLineWidth} in drawing dimmed states. It might be larger
than~$1$ to compensate optical effects when 
\textit{DimStateLineStyle} is \texttt{dashed} or \texttt{dotted}.

The style of dimmed states can be fixed by the command
\\\verb+\FixDimState{+\textit{LineStyle}\verb+}{+\textit{LineColor}\verb+}{+\textit{LineCoef}%
\verb+}{+\textit{LabelColor}\verb+}{+\textit{FillColor}\verb+}+.
\index{FixDimState}
\medskip

In mode \textsl{StateLineDouble} or when one uses the command \verb+\FinalState+,
two parameters give the width of each of both lines
and the distance between them. These values are coefficients that are
applied to \textsl{StateLineWidth}.

\begin{center}
\begin{tabular}{c|c|c|}
Parameter & Default value\\
\hline
\textsl{StateLineDblCoef} & \texttt{0.6} \\
\hline
\textsl{StateLineDblSep} & \texttt{0.4} \\
\hline
\end{tabular}
\end{center}
\index{StateLineDblCoef}\index{StateLineDblSep}

These values can be fixed by the command 
\verb+\FixStateLineDouble{+\textsl{Coef}\verb+}{+\textsl{Sep}\texttt{\}}.
\index{FixStateLineDouble}

%%%%%%%%%%%%%%%%%%%%%%
\subsection{Transition parameters}\label{sec.tra-par}
\paragraph{Edge adjustable parameters}
~ Each following parameter can be modified by the command
\verb+\Chg+\textit{Parameter}. The style value can be restored by \verb+\Rst+\textit{Parameter}
and one can set the style value with \verb+\Set+\textit{Parameter}.

\begin{center}
\begin{tabular}{c|c|c|}
Parameter & Default value & Possible values\\
\hline
\textit{EdgeLineStyle} & \texttt{solid} & \texttt{dashed}, \texttt{dotted}, \texttt{none}\\
\hline
\textit{EdgeLineWidth} & \texttt{1pt} &\\
\hline
\textit{EdgeLineColor} & \texttt{black} & many colours\\
\hline
\textit{EdgeLineDblStatus} & \texttt{false} &\texttt{true}\\
\hline
\textit{EdgeNodeSep} & \texttt{0pt} & \\
\hline
\textit{EdgeLabelColor} & \texttt{black} & many colours\\
\hline
\textit{EdgeLabelScale} & \texttt{1.7} &\\
\hline
\textit{EdgeLabelSep} & \texttt{3.5pt} & \\
\hline
\end{tabular}
\end{center}
\index{EdgeLineStyle}\index{EdgeLineWidth}\index{EdgeLineColor}
\index{EdgeLabelColor}\index{EdgeLabelScale}\index{EdgeLineDblStatus}
\index{EdgeLabelSep}\index{EdgeNodeSep}

As for states, the argument of \verb+\ChgEdgeLineWidth+ is
a coefficient applied to the style line width.

The parameter \textit{EdgeNodeSep} defines a distance between the 
begining, and the end, of a transition and the state it starts from 
and arrives to. The default value is \texttt{0pt}, that is, no 
distance at all; but it may be the case, especially when there is no 
line around the state that one may want to increase the distance 
between the transition and the label of the state.

The parameter \textit{EdgeLabelSep} controls the distance between the 
box that contains the label of the transition (built by \LaTeX) and 
the transition.


\paragraph{Edge preset parameters}
~
A first class of preset parameters deals with the definition of ``dimmed'' edges.

\begin{center}
\begin{tabular}{c|c|c|}
Parameter & Default value\\
\hline
\textit{DimEdgeLineStyle} & \texttt{solid} \\
\hline
\textit{DimEdgeLineColor} & \texttt{gray} \\
\hline
\textit{DimEdgeLineCoef} & \texttt{1.2} \\
\hline
\textit{DimEdgeLabelColor} & \texttt{gray}\\
\hline
\end{tabular}
\end{center}
\index{DimEdgeLineStyle}\index{DimEdgeLineColor}\index{DimEdgeLineCoef}\index{DimEdgeLabelColor}

The style of dimmed edges can be fixed by the command
\\\verb+\FixDimEdge{+\textit{LineStyle}\verb+}{+\textit{LineColor}\verb+}{+\textit{LineCoef}%
\verb+}{+\textit{LabelColor}\verb+}+.
\index{FixDimEdge}
\medskip

The style of the border of edges (in mode \verb+\EdgeBorder+),
is defined by two parameters~:
\begin{center}
\begin{tabular}{c|c|c|}
Parameter & Default value\\
\hline
\textsl{EdgeLineBorderCoef} & \texttt{2} \\
\hline
\textsl{EdgeLineBorderColor} & \texttt{white} \\
\hline
\end{tabular}
\end{center}
\index{EdgeLineBorderCoef}\index{EdgeLineBorderColor}

\verb+\EdgeLineBorderCoef+ is a coefficient applied to \verb+\EdgeLineWidth+,
that gives the width of the border.\\The style of borders of edges can
be fixed by the command \verb+\FixEdgeBorder{+\textsl{Coef}\verb+}{+\textsl{Color}\texttt{\}}.
\index{FixEdgeBorder}
\medskip

The width of the double line of (in mode \textsl{EdgeLineDouble}),
is defined by two parameters, that give the width of each of both lines
and the distance between them. These values are coefficients that are
applied to \textsl{EdgeLineWidth}.

\begin{center}
\begin{tabular}{c|c|c|}
Parameter & Default value\\
\hline
\textsl{EdgeLineDblCoef} & \texttt{0.5} \\
\hline
\textsl{EdgeLineDblSep} & \texttt{0.6} \\
\hline
\end{tabular}
\end{center}
\index{EdgeLineDblCoef}\index{EdgeLineDblSep}

These values can be fixed by the command \verb+\FixEdgeLineDouble{+\textsl{Coef}\verb+}{+\textsl{Sep}\texttt{\}}.
\index{FixEdgeLineDouble}

\subsection{Advanced parameters}
These parameters can be changed by using the usual command \verb+\renewcommand+
(or \verb+\setlength+, when the parameter is a length).

As it has already be mentionned, the preset commands that set
the diameter of states control
parameters for drawing edges (the length of initial and final arrows or
the curvature of loops). Their values are:
\begin{center}
\begin{tabular}{c|c|c|c|}
\textit{X} & \verb+\ArrowOn+\textit{X} & \verb+\LoopOn+\textit{X} & \verb+\CLoopOn+\textit{X}\\
\hline
\verb+LargeState+ & \texttt{1.3} & \texttt{5.8} & \texttt{6}\\
\hline
\verb+MediumState+ & \texttt{1.5} & \texttt{5.8} & \texttt{8}\\
\hline
\verb+SmallState+ & \texttt{1.7} & \texttt{5.8} & \texttt{12}\\
\hline
\verb+VariableState+ &  & \texttt{5.1} & \texttt{5.2}\\
\hline
\verb+VerySmallState+ & \texttt{5} & \texttt{15} & \\
\hline
\end{tabular}
\end{center}
\index{ArrowOnState}\index{LoopOnState}\index{CloopOnState}

\textit{ArrowOnX} is the coefficient that is applied to the radius of the state
to obtain the length of initial or final arrows. \textit{LoopOnX} and 
\textit{CLoopOnX}
is the eccentricity of loops and closed loops respectively.

Another parameter determines whether the dimension of states
refer to the middle, the inner or the outer side of the boundary.
There exist one parameter for classical states and one
for states with double line:
\begin{center}
\begin{tabular}{c|c|}
Parameter & Default value \\
\hline
\verb+\StateDimen+ & \texttt{outer} \\
\hline
\verb+\StateDblDimen+ & \texttt{middle} \\
\hline
\end{tabular}
\end{center}\index{StateDimen}{\index{StateDblDimen}
The possible values are \texttt{inner}, \texttt{outer} or \texttt{middle}.

\noi 
\hspace*{-\jsIndent}
\begin{minipage}[c]{\ColFigur}%
\par\vspace*{0mm}% pour l'alignement sur le haut
\begin{center}
\ShowGrid
\VCDraw{%
\begin{VCPicture}{(-1,-1)(6,2)}
\ChgStateLineWidth{5}
\FixStateDiameter{1cm}
\renewcommand{\StateDimen}{inner}
\State{(0.5,0.5)}{I}
\renewcommand{\StateDimen}{middle}
\State{(2.5,0.5)}{M}
\renewcommand{\StateDimen}{outer}
\State{(4.5,0.5)}{O}
\end{VCPicture}}
\HideGrid
\end{center}
\end{minipage}%
\hspace*{1.2em}% 
\begin{minipage}[c]{\ColSource}
\setlength{\parindent}{\parindenttemp}%
\par\vspace*{0mm}% pour l'alignement sur le haut
\footnotesize
\begin{verbatim}
\ShowGrid
\VCDraw{%
\begin{VCPicture}{(-1,-1)(6,2)}
\ChgStateLineWidth{5}
\FixStateDiameter{1cm}
\renewcommand{\StateDimen}{inner}
\State{(0.5,0.5)}{I}
\renewcommand{\StateDimen}{middle}
\State{(2.5,0.5)}{M}
\renewcommand{\StateDimen}{outer}
\State{(4.5,0.5)}{O}
\end{VCPicture}}
\HideGrid
\end{verbatim}
\end{minipage}% 

%%%%%%%%%%%%%%%%%%%%%%
\paragraph{Transition adjustable geometric parameters}
~
All these parameters are adjustable. That means there
exist commands as \verb+\Set+\textsl{Parameter}, \verb+\Chg+\textsl{Parameter}
and \verb+\Rst+\textsl{Parameter}.

Arcs between states are characterized by some adjustable parameters:

\begin{center}
\begin{tabular}{c|c|c|c|}
\textsl{X} & \textsl{XAngle} & \textsl{XCurvature} & \textsl{XOffset} \\
\hline
\textsl{Arc} & \texttt{15} & \texttt{0.8} & \texttt{1}pt\\
\hline
\textsl{LArc} & \texttt{30} & \texttt{0.8} & =\textsl{ArcOffset}\\
\hline
\end{tabular}
\end{center}
\index{ArcAngle}\index{ArcCurvature}\index{ArcOffset}

\medskip
Except the curvature that depends on the size of states,
parameters that define loop,
the degree of opening and the offset, are adjustable parameters.

\begin{center}
\begin{tabular}{c|c|c|}
\textsl{X} & \textsl{XAngle} (degrees) & \textsl{XOffset} \\
\hline
\textsl{Loop} & \texttt{30} & \texttt{0}pt\\
\hline
\textsl{CLoop} & \texttt{22} & =\textsl{LoopOffset}\\
\hline
\textsl{LoopVar} & \texttt{28} & \texttt{0.7}pt\\
\hline
\end{tabular}
\index{LoopAngle}\index{LoopOffset}
\VCDraw{%
\begin{VCPicture}{(-3,-3)(3,3)}
\State{(0,0)}{A}
\Point{(0,0)}{B}
\Point{(-1,1.73)}{C}
\Point{(1,1.73)}{D}
\Point{(1,-1.73)}{E}
\Point{(1.09,-1.68)}{F}
\Point{(-.42,.92)}{G}
\Point{(1.86,-1.23)}{H}
\Point{(0.14,-2.23)}{I}
\ChgEdgeLineWidth{.5}
\EdgeR{H}{F}{} \EdgeR{I}{E}{}
\ChgEdgeArrowStyle{<->}
\ArcL[.5]{C}{D}{2\times\text{Angle}}
\ChgEdgeArrowStyle{-}
\EdgeR{E}{F}{\text{Offset}}
\DimEdge
\ChgEdgeLineWidth{.1}
\ChgEdgeLineStyle{dashed}
\EdgeL{E}{C}{} \EdgeL{B}{D}{} \EdgeL{F}{G}{}
\RstEdge
\LoopXL{.5}{90}{A}{30}{.1cm}{15}{}
%\LoopXL{.5}{A}{90}{30}{0}{15}{}
\end{VCPicture}%
}
\end{center}

\medskip
\noi
Likewise, one can modify the style of arrows, that are defined by 
three adjustable parameters~:


\begin{center}
\begin{tabular}{c|c|}
Parameter & Default value \\
\hline
\textsl{EdgeArrowStyle} & \texttt{->} \\
\hline
\textsl{EdgeArrowWidth} & \texttt{5}pt \\
\hline
\textsl{EdgeArrowLengthCoef} & \texttt{1.4}  \\
\hline
\textsl{EdgeArrowInsetCoef} & \texttt{0.1}  \\
\hline
\end{tabular}
\VCDraw{%
\begin{VCPicture}{(-4,-2)(4,2)}
\pspolygon*(-1,-1)(0,1)(1,-1)(0,-.3)
\ChgEdgeArrowStyle{<->}
\Point{(-1.5,-1)}{A}
\Point{(-1.5,1)}{Ah}
\Point{(1.5,-1)}{B}
\Point{(1.5,-.3)}{Bh}
\Point{(-1,-1.5)}{C}
\Point{(1,-1.5)}{Ch}
\EdgeL{A}{Ah}{\text{Length}}
\EdgeR{B}{Bh}{\text{Inset}}
\EdgeR{C}{Ch}{\text{Width}}
\end{VCPicture}
}
\end{center}
\index{EdgeArrowStyle}\index{EdgeArrowWidth}%
\index{EdgeArrowLengthCoef}\index{EdgeArrowInsetCoef}

The {\em length} and the {\em inset} of arrows are given by 
coefficients that are applied to the {\em width}. 
There exists some other (non adjustable) parameters to define arrows 
in special cases: 
\verb+\EdgeArrowRevStyle+ is equal to \texttt{<-}, to draw transition 
in reverse mode; for edges with double lines, the size of 
arrows is a little bit more large. 
It is defined by \verb+\EdgeDblArrowWidth+
and \verb+\EdgeDblArrowLengthCoef+ that are respectively equal to 
\texttt{5.5}pt and \texttt{1.7}. 

\medskip
\noi
Zigzag edges are characterized by some adjustable parameters:

\begin{center}
\begin{tabular}{c|c|}
Parameter & Default value \\
\hline
\textsl{ZZSize} & \texttt{0.9}cm \\
\hline
\textsl{ZZLineWidth} & \texttt{1.7}  \\
\hline
\end{tabular}
\end{center}
\index{ZZSize}\index{ZZlineWidth}

\textsl{ZZSize} is the apparent diameter of the zigzag and \textsl{ZZLineWidth}
is a coefficient that is applied to \textsl{EdgeLineWidth} and gives
the width of the line.

\paragraph{Transition preset parameters}
~
The default position of labels on transitions depends on the type of transition.
For instance, on edges, this position is given by the variable \verb+\EdgeLabelPosit+
that is equal to $0.45$. The first column of the following array gives these values
for every transition.

\begin{center}
\begin{tabular}{c|c|c|}
\textsl{X} & \verb+\+\textsl{X}\texttt{LabelPosit} & \verb+\+\textsl{X}\texttt{LabelRevPosit} \\
\hline
\texttt{Edge} & \texttt{.45} & \texttt{.55}\\
\hline
\texttt{Arc} & \texttt{.4} & \texttt{.6}\\
\hline
\texttt{LArc} & \texttt{.4} & \texttt{.6}\\
\hline
\texttt{Loop} & \texttt{.25} & \texttt{.75}\\
\hline
\texttt{CLoop} & \texttt{.25} & \texttt{.75}\\
\hline
\texttt{Init} & \texttt{.1} & \texttt{.9}\\
\hline
\texttt{Final} & \texttt{.9} & \texttt{.1}\\
\hline
\end{tabular}
\end{center}

When the reverse mode is on, the position of label is given by
variables that correspond to the second column, as \verb+\EdgeLabelRevPosit+ for instance.
For the same type of transition, the sum of both variables
that defined position for straight and reverse arrows should
be equal to $1$ (if one wants to keep the same appearance when using
the \verb+ReverseArrow+ command).

These parameters can be modified with a \verb+\renewcommand+.

%%%%%%%%%%%%%%%%%%%%%%
\section{Using ``plain'' \PSTricks commands}
\label{sec.pla}
\VCSG is a set of macros based on \PSTricks. It is therefore
fully compatible with \PSTricks instructions.
Actually, the \verb+VCPicture+ environment is a specialization
of the \verb+pspicture+ environment; more, every state in \VCSG
is a \PSTricks node and every transition (loop, edge, arc)
is a node connection.

The following example shows how \VCSG commands can be mixed
with plain \PSTricks commands. For further explanation on
\PSTricks, we refer to~\cite{Girou94,GRM97,GRMRV07,Voss07}.


\noi 
\hspace*{-\jsIndent}
\begin{minipage}[t]{\ColFigur}%
\par\vspace*{0mm}% pour l'alignement sur le haut
\bigskip 
\begin{center}
\scalebox{\VCScale}{%
%
\begin{VCPicture}{(-7.5,-1)(3,2)}
\rput(-5,0){\rnode{X}{\psframebox{$\bigoplus$}}}
\rput(.5,0){\rnode{Y}{\psframebox{$\bigoplus$}}}
\State[1]{(-3.5,0)}{A}
\State[0]{(-1,0)}{B} 
\State[0]{(2,0)}{C}
\Point{(-7,0)}{S}
\EdgeL{S}{X}{0011}
\EdgeL{X}{A}{}\EdgeL{A}{B}{}\EdgeL{B}{Y}{}\EdgeL{Y}{C}{}
\ncangles[angleB=90, armB=1cm]{C}{X}
\ncangles[angleB=90, armB=1cm]{C}{Y}
\end{VCPicture}}
\\The shif register for $1+x^2+x^3$
\end{center}
\end{minipage}%
\hspace*{1.2em}% 
\begin{minipage}[t]{\ColSource}
\setlength{\parindent}{\parindenttemp}%
\par\vspace*{0mm}% pour l'alignement sur le haut
\footnotesize
\begin{verbatim}
\begin{VCPicture}{(-7.5,-1)(3,2)}
\rput(-5,0){\rnode{X}{\psframebox{$\bigoplus$}}}% pstricks node
\rput(.5,0){\rnode{Y}{\psframebox{$\bigoplus$}}}% pstricks node
\State[1]{(-3.5,0)}{A} \State[0]{(-1,0)}{B} \State[0]{(2,0)}{C}
\Point{(-7,0)}{S}
\EdgeL{S}{X}{0011}
\EdgeL{X}{A}{} \EdgeL{A}{B}{} \EdgeL{B}{Y}{} \EdgeL{Y}{C}{}
\ncangles[angleB=90, armB=1cm]{C}{X}% pstricks connection
\ncangles[angleB=90, armB=1cm]{C}{Y}% pstricks connection
\end{VCPicture}
\end{verbatim}
\normalsize
\end{minipage}% 
%

% %%%%%%%%%%%%%%%%%%%ù
% \section{Using Beamer document class}
% \VCSG can be used with Beamer document class to design slides.
% The document class
% should be called with {\tt xcolor=pst,dvips} options:
% 
% \verb+\documentclass[xcolor=pst,dvips]{beamer}+.

\clearpage

% \vspace*{5cm}

\appendix


\begin{center}
    {\Large \textsc{appendix}}
\end{center}

\section{Version history}

A first version of the package, which would be numbered 0.1, was 
called \emph{Vaucanson} and used by the authors in their papers 
between 2000 and 2003.
But, at the same time, the authors were involved, in cooperation with
the LRDE at EPITA, in the design of a \texttt{C++} platform for 
programming automata, for which it was decided that there would be no 
better name than Vaucanson as well.
We thus turned the name of the package for drawing automata to \VCSG.
% The name of every \textbf{file} of the library has been changed accordingly.
% No name of \textbf{command} in the package has been changed.

Version 0.2 was issued in May 2003 and made available on the web 
together with this user's manual first version.
A series of bug corrections and slight improvements made us switch at 
some point to version 0.3.

Just before sending the package to CTAN, we undertook a full review of 
the code and updated the manual accordingly. 
We call it version 0.4.
It is the last release before a completely new version of the 
package, with many more options , which is currently under work.


\clearpage 

\section{Style files}

\subsection{\texttt{slides} style}


{\footnotesize
\begin{verbatim}
%%%%%%%%%%%%%%%%%%%%%%%%%%%%%%%
%%
%% Package `Vaucanson-G'  version 0.4
%%
%% This is file `VCPref-slides'.
%%
%%%%%%%%%%%%%%%%%%%%%%%%%%%%%%%
%%% Scales  --- slides settings
%%%%%%%%%%%%%%%%%%%%%%%%%%%%%%%
\renewcommand{\LargeScale}{1.5}
\renewcommand{\MediumScale}{1.16}
\renewcommand{\SmallScale}{0.92}
\renewcommand{\TinyScale}{0.75}
%%%%%%%%%%%%%%%%%%%%%%%%%%%%%%
% State aspect
%%%%%%%%%%%%%%%%%%%%%%%%%%%%%%%%%
\SetStateLineColor{blue}
\SetStateLineWidth{2pt}
\SetStateFillColor{Cyan}
\SetStateLabelColor{Sepia}
\FixDimState{solid}{YellowOrange}{1}{YellowOrange}{white}
%%%%%%%%%%%%%%
% Edge aspect
%%%%%%%%%%%%%%
\SetEdgeLineColor{Mahogany}
\SetEdgeLineWidth{1.2pt}
\SetEdgeLabelColor{OliveGreen}
%%% Dimmed edges
\FixDimEdge{dashed}{1.2}{Apricot}{Apricot}
%%%%%%%%%%%%%%%%%%%%%%%%%%%%%%%
\end{verbatim}}
\newpage 

\subsection{\texttt{beamer} style}


{\footnotesize
\begin{verbatim}
%%%%%%%%%%%%%%%%%%%%%%%%%%%%%%%
%%
%% Package `Vaucanson-G'  version 0.4
%%
%% This is file `VCPref-slides'.
%%
%%%%%%%%%%%%%%%%%%%%%%%%%%%%%%%
%%% Scales  --- slides settings
%%%%%%%%%%%%%%%%%%%%%%%%%%%%%%%
\renewcommand{\LargeScale}{0.85}       
\renewcommand{\MediumScale}{0.6}       
\renewcommand{\SmallScale}{0.4}		
\renewcommand{\TinyScale}{0.30}		
%%%%%%%%%%%%%%%%%%%%%%%%%%%%%%
% State aspect
%%%%%%%%%%%%%%%%%%%%%%%%%%%%%%%%%
\SetStateLineColor{blue}
\SetStateLineWidth{2pt}
\SetStateFillColor{Cyan}
\SetStateLabelColor{Sepia}
\FixDimState{solid}{YellowOrange}{1}{YellowOrange}{white}
%%%%%%%%%%%%%%
% Edge aspect
%%%%%%%%%%%%%%
\SetEdgeLineColor{Mahogany}
\SetEdgeLineWidth{1.2pt}
\SetEdgeLabelColor{OliveGreen}
%%% Dimmed edges
\FixDimEdge{dashed}{1.2}{Apricot}{Apricot}
%%%%%%%%%%%%%%%%%%%%%%%%%%%%%%%
\end{verbatim}}
\newpage 

\subsection{\texttt{mystyle} style}

This file is given only on the purpose that it can be edited by the 
user and called by the option \texttt{mystyle} of the wrapper(and this is the 
reason why the \emph{name} \verb+VCPref-mystyle.tex+ 
of this file should not be changed).

The actual values given in this file as an example provide a style 
which mimics the way automata are drawn with GasTeX \cite{Gast} as 
can be witnessed below.

\noi 
\hspace*{-\jsIndent}
\begin{minipage}[c]{\ColFigur+.5cm}%
\par\vspace*{0mm}% pour l'alignement sur le haut
\begin{center}
\input{VCPref-mystyle}
\VCDraw{%
\begin{VCPicture}{(0,-2)(6,2)}
% states
\State[A]{(0,0)}{A} \State[B]{(2,-2)}{B}
\FinalState[C]{(4,2)}{C} \State[D]{(6,0)}{D}
%initial-final
\Initial{A} \Final[s]{B}
% transitions 
\ArcR{A}{B}{a} \ArcL{A}{C}{b}
\LArcR{B}{D}{c} \LArcL{C}{D}{d}
\ArcL{A}{B}{a'} \EdgeLineDouble \EdgeR{A}{C}{b'}
\EdgeLineSimple \EdgeL{B}{D}{c'} \LArcR{C}{D}{d'}
%
\end{VCPicture}%
}%
\end{center}
\end{minipage}%
\hspace*{1.2em}% 
\begin{minipage}[c]{\ColSource-.5cm}
\setlength{\parindent}{\parindenttemp}%
\par\vspace*{0mm}% pour l'alignement sur le haut
\footnotesize
\begin{verbatim}
\input{VCPref-mystyle}
\VCDraw{%
\begin{VCPicture}{(0,-2)(6,2)}
% states
\State[A]{(0,0)}{A} \State[B]{(2,-2)}{B}
\FinalState[C]{(4,2)}{C} \State[D]{(6,0)}{D}
%initial-final
\Initial{A} \Final[s]{B}
% transitions 
\ArcR{A}{B}{a} \ArcL{A}{C}{b}
\LArcR{B}{D}{c} \LArcL{C}{D}{d}
\ArcL{A}{B}{a'} \EdgeLineDouble \EdgeR{A}{C}{b'}
\EdgeLineSimple \EdgeL{B}{D}{c'} \LArcR{C}{D}{d'}
%
\end{VCPicture}}
\end{verbatim}
\normalsize
\end{minipage}% 

\bigskip\bigskip
{\footnotesize
\begin{verbatim}
%%%%%%%%%%%%%%%%%%%%%%%%%%%%%%%
%%
%% Package `Vaucanson-G'  version 0.4
%%
%% This is file `VCPref-mystyle'.
%%  
%%%%%%%%%%%%%%%%%%%%%%%%%%%%%%%
%%% Scales  --- 
%%%%%%%%%%%%%%%%%%%%%%%%%%%%%%%
\renewcommand{\LargeScale}{1.2}         %float : argument of a \scalebox
\renewcommand{\MediumScale}{1}          %float
\renewcommand{\SmallScale}{.7}		%float
\renewcommand{\TinyScale}{0.5}		%float
%%%%%%%%%%%%%%%%%%%%%%%%%%%%%%%
%%% State parameters  --- Default settings
%%%%%%%%%%%%%%%%%%%%%%%%%%%%%%%
\setlength{\LargeStateDiameter}{1.2cm}		%length
\setlength{\MediumStateDiameter}{.8cm}		%length
\setlength{\SmallStateDiameter}{.6cm}		%length
\renewcommand{\StateDblDimen}{outer}
%%%%%%%%%%%%%%%%%%%%%%%%%%%%%%%
% State aspect
%%%%%%%%%%%%%%%%%%%%%%%%%%%%%%%%%
\SetStateLineWidth{.14mm}		%% length
\SetStateFillStatus{none}		%% aspect
\SetStateFillColor{black}		%% color
\SetStateLabelScale{1}                  %% float
\FixStateLineDouble{2}{5}               %% Double style: 
%%%%%%%%%%%%%%
% Edge aspect
%%%%%%%%%%%%%%
\SetEdgeLineWidth{.14mm}			%% length
\SetEdgeLabelColor{black}		%% color
\SetEdgeLabelScale{1}		        %% float
\FixEdgeLineDouble{1.5}{2}		%% float : 
%%% arrows
\SetEdgeArrowWidth{1.03mm}			%width of the edge arrow
\SetEdgeArrowLengthCoef{1.37}                   %float : 
\setlength{\EdgeDblArrowWidth}{1.3mm}		%width : 
\renewcommand{\EdgeDblArrowLengthCoef}{1.09}	% 
\SetEdgeArrowInsetCoef{0}		%float : coef*\EdgeArrowSizeDim
%%%%%%%%%%%%%%%%%%%%%%%%%%%%%%%
% Arc geometry
%%%%%%%%%%%%%%%%%%%%%%%%%%%%%%%
\SetArcAngle{17}			%% int (degree)
\SetLArcAngle{30}			%% int (degree)
\SetArcCurvature{0.7}			%% float
\SetArcOffset{1pt}			%% length
%%%%%%%%%%%%%%%%%%%%%%%%%%%%%%%
% Loop geometry
%%%%%%%%%%%%%%%%%%%%%%%%%%%%%%%
\renewcommand{\LoopOnLargeState}{5.5} 		%float
\renewcommand{\LoopOnMediumState}{7}		%float : curvature
\renewcommand{\LoopOnSmallState}{9} 		%float
\renewcommand{\LoopOnVariableState}{4.5} 		%float
%%%%%%%%%%%%%%%%%%%%%%%%%%%%%%%
%%% Edge labels positionning
%%%%%%%%%%%%%%%%%%%%%%%%%%%%%%%
\renewcommand{\EdgeLabelPosit}{.5}   %per cent 
\renewcommand{\EdgeLabelRevPosit}{.5}
\renewcommand{\ArcLabelPosit}{.5}
\renewcommand{\ArcLabelRevPosit}{.5}
\renewcommand{\LArcLabelPosit}{.5}
\renewcommand{\LArcLabelRevPosit}{.5}
\renewcommand{\LoopLabelPosit}{.5}
\renewcommand{\LoopLabelRevPosit}{.5}
\renewcommand{\CLoopLabelPosit}{.5}
\renewcommand{\CLoopLabelRevPosit}{.5}
%%%%%%%%%%%%%%%%%%%%%%%%%%%%%%%
%%% Initial states parameters
%%%%%%%%%%%%%%%%%%%%%%%%%%%%%%%
\renewcommand{\ArrowOnMediumState}{1}		%float 
\renewcommand{\ArrowOnSmallState}{1} 		%float
\renewcommand{\ArrowOnLargeState}{1}		%float
%%%%%%%%%%%%%%%%%%%%%%%%%%%%%%%
\end{verbatim}}

\clearpage 

\begin{thebibliography}{99}
\bibitem{Girou94}
{\sc D.~Girou}
\newblock Pr\'esentation de PSTricks,
\newblock {\em Cahier GUTenberg} \textbf{16} (1994) 21-70.
%  \texttt{http://www.tug.org/applications/PSTricks/}

\bibitem{Gast}
{\sc P.~Gastin}
\newblock {\em GasTeX: Graphs and Automata Simplified in TeX}.\\
\newblock \texttt{http://www.lsv.ens-cachan.fr/~gastin/gastex/gastex.html}

\bibitem{GRM97}
{\sc M.~Goossens, S.~Rahtz and F.~Mittelbach}
\newblock {\em The LaTeX Graphics Companion}.
\newblock Addison Wesley, 1997.

\bibitem{GRMRV07}
{\sc M.~Goossens, S.~Rahtz, F.~Mittelbach, D. Roegel, and H. Voss}
\newblock {\em The LaTeX Graphics Companion, 2nd Edition}.
\newblock Addison Wesley, 2007.

\bibitem{Sak01}
{\sc J. Sakarovitch},
\newblock {\em El\'ements de th\'eorie des automates}.
\newblock Vuibert, 2003. 
\newblock Eng. trans. {\em Elements of Automata Theory}.
\newblock Cambridge University Press, to appear.


\bibitem{Voss07}
{\sc H. Voss},
\newblock {\em PSTricks -- Grafik f\"ur TeX und LaTeX}.
\newblock Lehmanns/Dante e.V., 4th ed. 2007. 

\end{thebibliography}

Some examples of articles or slides that contain automata drawn with \VCSG
can be found at the following URLs:
\begin{itemize}
\item \texttt{http://www.enst.fr/${\scriptstyle\pmb{\sim}}$jsaka/}
\item \texttt{http:///igm.univ-mlv.fr/${\scriptstyle\pmb{\sim}}$lombardy/}
\end{itemize}

\printindex


%\newpage
%\input{VCManual_tab.tex}
\end{document}
%%%%%%%%%%%%%%%%%%%%%
